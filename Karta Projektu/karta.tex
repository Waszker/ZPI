\documentclass{article}
\usepackage[MeX]{polski}
\usepackage[utf8]{inputenc}
\author{Fundacja nowoczesna szkoła\\\\
Honorata Rosłanowska \\
Marcin Wardziński\\
Kacper Sarnacki \\
Łukasz Dragan \\
Kacper Trojanowski \\
Mateusz Flis \\
Michał Grabowski \\
Piotr Waszkiewicz}
\title{System polis ubezpieczeniowych}

\begin{document}
\maketitle
\newpage
\tableofcontents
\newpage

\section{Uzasadnienie i cel projektu}
Przyczyną stworzenia zamówienia i uruchomienia projektu jest brak wystarczającej wydajności, oraz ograniczona funkcjonalność obecnie stosowanego systemu. \\
Celem projektu jest stworzenie systemu polis ubezpieczeniowych, który usprawni działanie firmy poprzez zwiększenie dostępności oferowanych produktów oraz przyspieszenie ich sprzedaży. Rozwiązanie będzie podobne do już istniejącego, o rozszerzonej funkcjonalności i zwiększonej wydajności, co przyczyni się do zwiększenia przychodów dla firmy. Docelowo system ten ma zastąpić obecnie działające rozwiązanie.

\section{Cele projektu, miary sukcesu}
\begin{itemize}
	\item System umożliwia opcję kalkulacji składki. Czas kalkulacji nie może przekraczać 5 sekund, przy niepełnej kalkulacji, czas ten nie może przekraczać 1 sekundy
	\item Założenie polisy przy 100 użytkownikach nie powinno przekraczać 3 sekund
	\item 5/10 wybranych przez Zamawiającego osób uzna interfejs graficzny za spełniający warunki zamówienia
	\item System spełnia wymogi dotyczące wydajności opisane w specyfikacji
\end{itemize}

\section{Wymagania}
\begin{itemize}
	\item System umożliwia tworzenie ofert oraz zawieranie polis
	\item Każda polisa ma możliwość poprawnej zmiany, rozwiązania lub wznowienia
	\item Pracownicy mają możliwość przeglądania i zarządzania polisami ubezpieczeniowymi
	\item Użytkownicy mają możliwość rejestracji oraz zawierania polis poprzez aplikacje mobilne na systemach iOS i Androd oraz przez stronę internetową przy użyciu przeglądarek
	\item System posiada moduł kalkulatora
	\item System umożliwia poprawne dokonywanie płatności poprzez aplikację
	\item System umożliwia integrację z systemami zewnętrznymi
	\item Do systemu stworzona jest dokumentacja
	\item W systemie wyodrębnione są role użytkowników
	\item System posiada dane pochodzące ze starego systemu
	
\end{itemize}

\section{Zakres projektu}
\begin{itemize}
	\item Wytworzenie oprogramowania
	\item Migracja danych
	\item Instalacja i uruchomienie systemu
	\item Szkolenia dla pracowników - przygotowanie kursów e-learningowych
	\item Wsparcie serwisowe
\end{itemize}

\section{Produkty projektu}
\begin{itemize}
	\item Aplikacja webowa dla pracowników oraz dla klientów firmy
	\item Aplikacja mobilna dla klientów firmy.
	\item Baza danych
\end{itemize}

\section{Opis ryzyka}
\begin{itemize}
	\item Dane w istniejącej bazie danych będą niekompletne/niespójne lub wymagać będą czasochłonnej konwersji, co skutkuje spowolnieniem implementacji nowego systemu oraz możliwą koniecznością skorzystania z usług eksperta
	\item Ryzyka związane z integracją z wieloma serwisami zewnętrznymi:
	\begin{itemize}
		\item braki w dokumentacji mogą spowolnić integrację
		\item w wyniku niespodziewanych zmian w interfejsie programistycznym serwisu, konieczna może być ponowna implementacja funkcjonalności
		\item ograniczenia transferu danych nałożone przez dostawców usług sprawiają, iż wydajność systemu będzie niższa niż oczekiwano
	\end{itemize}
	\item Brak zachowania ciągłości w dostępności wszystkich specjalistów zgodnie z wymaganiami Zleceniodawcy może skutkować poniesieniem kar pieniężnych
	\item Brak specjalistycznej wiedzy dotyczącej tematyki polis. Wiąże się z tym ryzyko spowolnienia implementacji funkcjonalności oraz konieczność skorzystania z usług doradczych osób z zewnątrz	
\end{itemize}

\section{Harmonogram}
\begin{itemize}
	\item Przygotowanie dokumentacji technicznej
	\item Implementacja
	\item Wdrożenie systemu
	\item Testy
	\item Migracja danych
	\item Szkolenia dla pracowników - kursy e-learningowe
	\item Zdanie gotowego rozwiązania	
\end{itemize}

\section{Budżet}
\begin{itemize}
	\item Koszt wytworzenia oprogramowania: 370 000 PLN
	\item Koszt przygotowania kursów e-learningowych: 60 000 PLN
	\item Koszt rocznego utrzymania systemu: 200 000 PLN
\end{itemize}

\section{Warunki odbioru}
\begin{itemize}
	\item Wraz z oddaniem projektu przekazana zostanie cała dokumentacja systemu
	\item Złożenie raportu z przygotowanych testów wraz z dokumentacją
	\item Certyfikat zgodności z zewnętrznymi serwisami
	\item Zaakceptowanie protokołu opisującego przeprowadzoną migrację danych
	\item Podpisanie dokumentu podsumowującego przeprowadzone szkolenia
	\item Podpisanie protokołu zdawczego	
\end{itemize}

\section{Menadżer projektu}
\begin{itemize}
	\item ŁD - lider zespołu programistycznego
	\begin{itemize}
		\item może wnioskować o zmianę budżetu
		\item może zwalniać lub zatrudniać pracowników
		\item odpowiada za projekt
	\end{itemize}	
\end{itemize}

\section{Sponsor}
\begin{itemize}
	\item PW - kierownik działu IT
	\begin{itemize}
		\item może zmieniać budżet
		\item może anulować projekt
		\item może zmieniać harmonogram
		\item odpowiada za negocjacje	
	\end{itemize}
\end{itemize}

\end{document}
