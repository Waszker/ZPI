\documentclass{article}
\usepackage[MeX]{polski}
\usepackage[utf8]{inputenc}
\usepackage{array}
\usepackage{longtable}

\linespread{1.4}

\newcolumntype{L}[1]{>{\raggedright\let\newline\\\arraybackslash\hspace{0pt}}m{#1}}
\newcolumntype{C}[1]{>{\centering\let\newline\\\arraybackslash\hspace{0pt}}m{#1}}
\newcolumntype{R}[1]{>{\raggedleft\let\newline\\\arraybackslash\hspace{0pt}}m{#1}}


\author{Fundacja nowoczesna szkoła\\\\
Honorata Rosłanowska \\
Marcin Wardziński\\
Kacper Sarnacki \\
Łukasz Dragan \\
Kacper Trojanowski \\
Mateusz Flis \\
Michał Grabowski \\
Piotr Waszkiewicz}
\title{System polis ubezpieczeniowych}

\begin{document}
\maketitle
\newpage

\section{Plan zarządzania jakością}

Celem stworzenia dokumentu pełniącego rolę planu zarządzania jakością jest zapewnienie świadczenia lepszej jakości usług, oraz stworzenie sprawnie działającego produktu końcowego. Ponieważ na jakość przygotowanego rozwiązania ma wpływ wiele czynników, zostały one zawarte i opisane w poniższej tabeli. \\

\begin{center}
\begin{tabular}{|>{\centering\arraybackslash}m{0.2\textwidth}|>{\centering\arraybackslash}m{0.4\textwidth}|>{\centering\arraybackslash}m{0.4\textwidth}|}
	\hline 
	\textbf{Miernik jakości} & \textbf{Metoda oceny} & \textbf{Wyjaśnienie} \\
	\hline
	
	Wydajność &
	\begin{minipage}[t]{0.4\textwidth}
		\begin{itemize}
			\item Testy akceptacyjne \\
			\item Testy obciążeniowe \\
			\item Testy jednostkowe \\
		\end{itemize} 
	\end{minipage} &
	\begin{minipage}[t]{0.4\textwidth}
		System spełnia wymagania dotyczące wydajności określone w specyfikacji: czas założenia polisy przy 100 użytkownikach nie powinien przekraczać 3 sekund, czas kalkulacji składni nie powinien przekraczać 5 sekund (w przypadku niepełnej składki 1 sekundy).
	\end{minipage}\\
	\hline
	
	Jakość kodu &
	\begin{minipage}[t]{0.4\textwidth}
		\begin{itemize}
			\item Testy jednostkowe \\
			\item Konwencje formatowania \\
			\item Continuous integration \\
			\item Code review \\
		\end{itemize} 
	\end{minipage} &
	\begin{minipage}[t]{0.4\textwidth}
		Podczas pisania kodu, programiści stosować się będą do wcześniej ustalonych zasad formatowania i nazewnictwa zmiennych. Testy jednostkowe pisane w ramach projektu, zastosowanie continuous integration wraz z code review, dostarczą mierzalnych wartości dotyczących jakości stworzonego kodu.
	\end{minipage}\\
	\hline
\end{tabular}
\end{center}
	
\begin{center}
\begin{tabular}{|>{\centering\arraybackslash}m{0.2\textwidth}|>{\centering\arraybackslash}m{0.4\textwidth}|>{\centering\arraybackslash}m{0.4\textwidth}|}
	\hline 
	\textbf{Miernik jakości} & \textbf{Metoda oceny} & \textbf{Wyjaśnienie} \\	
	\hline
	
	Dokumentacja &
	\begin{minipage}[t]{0.4\textwidth}
		\begin{itemize}
			\item Udokumentowane klasy i metody publiczne \\
			\item Dokumentacja systemu \\
			\item Instrukcja instalacji \\
			\item Instrukcja obsługi \\
		\end{itemize} 
	\end{minipage} &
	\begin{minipage}[t]{0.4\textwidth}
		Dostarczony kod powinien być przejrzyście i schludnie napisany. Dodatkowo, każda publiczna klasa oraz ich metody powinny być opatrzone stosownym komentarzem tłumaczącym ich funkcję. Wraz z oddaniem systemu dostarczona powinna zostać odpowiednia instrukcja instalacji oraz obsługi a także dokument opisujący najważniejsze elementy systemu. Wszystkie instrukcje powinny zostać napisane w sposób prosty do zrozumienia, wykorzystując w tym celu schematy i obrazki.
	\end{minipage}\\
	\hline
	
	Bezpieczeństwo systemu &
	\begin{minipage}[t]{0.4\textwidth}
		\begin{itemize}
			\item Szyfrowane połączenia \\
			\item Bezpieczny system logowania \\
		\end{itemize} 
	\end{minipage} &
	\begin{minipage}[t]{0.4\textwidth}
		Każde połączenie z zewnętrznym serwisem realizowane jest przy użyciu protokołu HTTPS. Użytkownicy systemu muszą założyć w nim konto na które logują się przy użyciu loginu i hasła.
	\end{minipage}\\
	\hline
	
	
\end{tabular}
\end{center}

\begin{center}
\begin{tabular}{|>{\centering\arraybackslash}m{0.2\textwidth}|>{\centering\arraybackslash}m{0.4\textwidth}|>{\centering\arraybackslash}m{0.4\textwidth}|}
	\hline 
	\textbf{Miernik jakości} & \textbf{Metoda oceny} & \textbf{Wyjaśnienie} \\
	\hline
	Bezpieczeństwo danych &
	\begin{minipage}[t]{0.4\textwidth}
		\begin{itemize}
			\item Kopie zapasowe \\
			\item Kontrola dostępu do narzędzi administracyjnych \\
			\item 'Zasolenie' haszy haseł
			\item Audyt każdej tabeli
		\end{itemize} 
	\end{minipage} &
	\begin{minipage}[t]{0.4\textwidth}
		Każda poważna zmiana w systemie wymaga autoryzacji ze strony administratora. Kopie zapasowe bazy danych są trzymane na dwóch, różnych nośnikach i regularnie uaktualniane. Haszowanie samych haseł nie jest bezpiecznym rozwiązaniem, w związku z tym będziemy haszowali hasło + login. Każda zmiana dokonywana przez użytkowników będzie zapisywana w bazie danych, co umożliwia odtworzenie przebiegu wydarzeń.
	\end{minipage}\\
	\hline	
	
	Terminowość &
	\begin{minipage}[t]{0.4\textwidth}
		\begin{itemize}
			\item Realizacja etapów zgodnie z harmonogramem \\
			\item Dotrzymanie terminu oddania projektu \\
			\item Realizacja przedsięwzięcia w ramach przewidzianego budżetu \\
		\end{itemize} 
	\end{minipage} &
	\begin{minipage}[t]{0.4\textwidth}
		Projekt powinien być realizowany zgodnie z ustalonymi etapami, w ramach przyjętych ram czasowych i budżetowych.
	\end{minipage}\\
	\hline
	
\end{tabular}
\end{center}



\end{document}
