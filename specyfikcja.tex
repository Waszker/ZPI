\documentclass{article}
\usepackage[MeX]{polski}
\usepackage[utf8]{inputenc}
\author{Fundacja nowoczesna szkoła\\\\
Imiona i nazwiska\\członków zespołu} % TODO
\title{eSzkoła - system zarządzania placówką oświatową}

\begin{document}
\maketitle
\newpage
\tableofcontents
\newpage

\section{Opis projektu}
% TODO napisać opis projektu

\section{Cele projektu}
Głównym celem zamawianego produktu jest usprawienie działania placówki oświatowej pod kątem:
\begin{itemize}
    \item administracji
    \item obsługi dydaktyki
    \item komunikacji
    % \item integracji TODO
    % TODO ew. dodać obsługę administracji - rodzice dokonują wpłat
\end{itemize}

W ramach zamówienia przygotowany zostanie system informatyczny wraz ze wsparciem technicznym i usługą serwisową.
% pakiet serwisowy


\section{Misja organizacji}
Założeniem istnienia Fundacji Nowoczesna Szkoła jest ... % TODO

dzieci i rodzice mają poczuć się ważną częścią placówki oświatowej\\
jak facebook - ma się stać częścią naszego życia\\
możliwość komunikowania się wiadomościami\\
w myśl zasady \textit{ oświata - like it!}\\
organizowanie wycieczek, szkolenia rozwijające \\
komunikacja komputerowa dla starszych pracowników szkół

\section{Opis zakresu projektu}
% wdrożenie w godz nocnych by nie zaburzyć pracy
% precyzyjnie określić czas działania
% obsłużenie żądań 100 użytkowników w czasie


Projekt składa się z następujących modułów
\begin{itemize}
    \item moduł kadr
    \item moduł ocen
    \item moduł rodzica
    \item moduł dydaktyczny
\end{itemize}

Przedmiotem Umowy jest dostarczenie, wykonanie, instalacja i wdrożenie systemu \textit{eSzkoła}, obejmuje w szczególności:
%TODO dodać licencję
\begin{enumerate}
    \item wytworzenie, dostarczenie, zainstalowanie, skonfigurowanie i uruchomienie Systemu na zasobach sprzętowych wskazanych przez Zamawiającego, zgodnego z obowiązującym
    prawem;
    \item dostarczenie bazy danych zawierającej informacje dostarczone przez Zamawiającego w plikach zgodnych z formatem *.xls;
    \item skonfigurowanie i zintegrowanie dostarczanego Systemu do współpracy z systemami dzienniczka elektronicznego Librus i Vulcan
    a także zewnętrznego systemu kadr %TODO Flis team
\end{enumerate}

\subsection{Wymagania funkcjonalne}
\begin{enumerate}
    \item System pozwala zalogować się po podaniu poprawnego loginu i hasła
    \item W systemie możliwe jest założenie kont użytkowników różnego typu:
    \begin{itemize}
        \item \textbf{konto ucznia}
        \begin{itemize}
            \item Zawiera informacje o imieniu, nazwisku, adresie ucznia oraz klasie, do której należy
            \item Pozwala na dostęp do planów zajęć wszystkich klas w szkole
            \item Umożliwia sprawdzenie swoich ocen cząstkowych i semestralnych, średniej ważonej ze wszystkich przedmiotów
            \item Udostępnia podgląd kalendarium szkoły
            \item Pozwala zapisać się do koła naukowego
            \item Umożliwia dostęp do prywatnego kalendarza zawierającego informacje o sprawdzianach, kołach naukowych, do których uczeń jest zapisany i zastępstwach nauczycieli, z którymi uczeń ma zajęcia
            \item Daje dostęp do materiałów dydaktycznych zamieszczanych przez nauczycieli
        \end{itemize}
        \item \textbf{konto rodzica} 
        \begin{itemize}
            \item Jest przypisane do co najmniej jednego ucznia
            \item Ma dostęp do wszystkich danych, do których ma dostęp jego dziecko
            \item Umożliwia usprawiedliwienie nieobecności poprzez dodanie informacji tekstowej lub udostępnienie zeskanowanych zwolnień lekarskich
            \item Pozwala komunikować się z nauczycielami i innymi rodzicami poprzez wiadomości tekstowe
            \item Zawiera moduł opłat %TODO ?
        \end{itemize}
        %TODO: Czy pisać o liście przedmiotów prowadzonych przez nauczyciela?
        \item \textbf{konto nauczyciela} 
        \begin{itemize}
            \item Zawiera informacje o imieniu, nazwisku, adresie nauczyciela
            \item Jest przypisane do co najmniej jednego przedmiotu
            \item Zawiera listę wszystkich uczniów uczęszczających na prowadzone przedmioty, podzieloną według przedmiotów i klas szkolnych
            \item Pozwala komunikować się z rodzicami, innymi nauczycielami i administratorami poprzez wiadomości tekstowe
            \item Umożliwia wystawianie ocen cząstkowych i semestralnych z przedmiotów, do których jest przypisany nauczyciel
            \item Umożliwia dodawanie informacji o sprawdzianach
            \item Pozwala na dostęp do swojego planu zajęć oraz do wszystkich planów zajęć klas i nauczycieli
            \item Umożliwia udostępnianie materiałów dydaktycznych dowolnego typu wybranym klasom
        \end{itemize}
        \item \textbf{konto administratora}
        \begin{itemize}
             \item Pozwala na zakładanie nowych kont użytkowników i tworzenie powiązań między nimi (np. przypisanie konta ucznia do odpowiadającego mu rodzica)
             \item Daje możliwość dodawania, edycji i usuwania klas oraz zapisanych do nich uczniów 
             \item Umożliwia modyfikowanie planu zajęć
             \item Pozwala na dodawanie i usuwanie z kalendarza informacji o nieobecności nauczycieli oraz wycieczkach szkolnych
             \item Umożliwia zmianę ocen uczniów
         \end{itemize} 
        % TODO: Czy konto SUPERUSER jest nam niezbędne?
        %\item \textbf{konto administratora głównego \"SUPERUSER\"} - konto o nieograniczonych przywilejach
    \end{itemize}
    \item Materiały dydaktyczne są sposobem na komunikację między nauczycielem a uczniami. Są to pliki dowolnego typu udostępniane w aplikacji. Nauczyciel ma możliwość dodawania i usuwania ich, uczeń ma możliwośc dodawania. Nauczyciel ma możliwość udostępniania materiałów wszystkim uczniom należącym do klasy, którą uczy. Jednorazowo wrzucone dane nie powinny mieć wielkości przekraczającej 10 MB.
    \item Plan zajęć jest przypisany do klasy albo nauczyciela. Zawiera informacje o przedmiotach: godzinach ich rozpoczęcia i zakończenia, sali, w której się odbywają, nauczycielu prowadzącym zajęcia (w przypadku planu ucznia).
    \item Kalendarium zawiera informacje o wydarzeniach szkolnych. Jest uniwersalne dla szkoły. Zawiera informacje o apelach i uroczystościach szkolnych, a także dniach wolnych od zajęć dydaktycznych.
    \item Kalendarz ucznia - jest to zbiór wydarzeń spersonalizowany dla każdego ucznia. Zawiera informacje o zastępstwach nauczycieli, z którymi uczeń ma zajęcia, informacje o sprawdzianach (ich data oraz przedmiot) oraz o kołach naukowych, do których zapisany jest uczeń
    \item Klasa jest zbiorem uczniów, posiada swoją nazwę oznaczaną jedną dużą literą alfabetu łacińskiego. Ma przypisanego wychowawcę - dokładnie jednego nauczyciela. Każda klasa ma przypisany plan zajęć.

    \item System oferuje integrację z istniejącymi rozwiązaniami dzienniczków elektronicznych LIBRUS oraz VULCAN na poziomie wyświetlania, modyfikacji oraz synchronizacji dwustronnej wprowadzonych ocen (tzn. wszelkie zmiany wprowadzone w dowolnym systemie powinny być odzwierciedlone w każdym z nich)
    \item Każde konto uprawnione do wystawiania ocen może to zrobić tylko w ramach swoich przedmiotów i tylko dla uczniów uczęszczających na te zajęcia
    \item Każda ocena powinna być wartością całkowitoliczbową z zakresu 1-6 z ewentualnym modyfikatorem w postaci symbolu \"-\" lub \"+\" występującym raz na samym końcu wartości oceny, posiadać przypisaną wagę (domyślnie ustawioną jako 1), zawierać krótką informację o powodzie jej wystawienia nie dłuższą niż 60 znaków a także datę wystawienia.
    \item Oceny wprowadzone do systemu mogą być usuwane oraz modyfikowane. Modyfikacja może obejmować zmiany w dowolnej wartości poza datą wystawienia oceny
    \item Możliwe jest wyliczenie średniej ważonej na podstawie zgromadzonych przez ucznia ocen i wyświetlenie jej jako proponowanej oceny końcowej
    \item Konto nauczyciela posiada możliwość zaakceptowania proponowanej oceny końcowej wyliczonej przez system jak również wprowadzenie własnej wartości
    \item Konta rodzica oraz ucznia mogą obejrzeć wystawione oceny a także wszystkie związane z nimi informacje
\end{enumerate}
%TODO podział funkcjonalnych:
system kadr: Dragan, Flis
system kadr: Pan Szef
system ocen: MW, KT
dydaktyka: K i H spis sprawdzianów, materiały dydaktyczne, plan zajęć, kalendarz, opis kont: uczeń ma możliwość podejrzenia, admin ma coś innego, nauczyciel może zgłosić prośbę o urlop, wypłaty, eksport cv (nauczyciel), elektroniczne wrzucanie zwolnień, zmiana wypłat,


% dostarczenie bazy danych zawierającej informacje o planie zajęć, kołach naukowych oraz uroczystościach szkolnych dostarczonych przez Zamawiającego w formie elektronicznej zgodnej z formatem pliku *.xls;
\subsection{Wymagania niefunkcjonalne}
system ma działać w 99,9\% przez okres roku
% Marcin Kacper

\subsection{Sposób wdrożenia}
% Łukasz

\subsection{Warunki odbioru systemu}
\section{Utrzymanie systemu po wdrożeniu}
odporność na błędy, że działa przez jakiś okres czasu
%TODO warunki co do ludzi utrzymujących: wykształcenie, certyfikaty

\section{Dokumentacja systemu}
% PW
\section{Zasoby fundacji}
% TODO K & H

\section{Termin zakończenia}
% TODO K & H

\section{Kary umowne}
\end{document}
