\documentclass{article}
\usepackage[MeX]{polski}
\usepackage[utf8]{inputenc}
\author{Fundacja nowoczesna szkoła\\\\
Honorata Rosłanowska \\
Marcin Wardziński\\
Kacper Sarnacki \\
Łukasz Dragan \\
Kacper Trojanowski \\
Mateusz Flis \\
Michał Grabowski \\
Piotr Waszkiewicz}
\title{eSzkoła - system zarządzania placówką oświatową}

\begin{document}
\maketitle
\newpage
\tableofcontents
\newpage

\section{Opis projektu}
Projekt eSzkoła zakłada stworzenie wielomodułowego systemu do zarządzania placówką oświatową. W ramach rozwiązania przewidziana jest realizacja modułów dla dyrektora, sekretariatu, nauczycieli, uczniów i ich rodziców pozwalających ułatwić dostęp i zarządzanie istotnymi danymi. \\ \\
Głównym celem zamawianego produktu jest usprawienie działania placówki oświatowej pod kątem:
\begin{itemize}
    \item administracji
    \item obsługi dydaktyki
    \item komunikacji
\end{itemize}
W ramach zamówienia przygotowany zostanie system informatyczny wraz ze wsparciem technicznym i usługą serwisową.

\section{Misja organizacji i cele projektu}
Fundacja Nowoczesna Szkoła została założona w 2006 roku jako organizacja wspierająca rozwój szkolnictwa. Jej celem jest ułatwienie dostępu do materiałów dydaktycznych, zapewnienie rodzicom lepszego nadzoru nad uczącym się dzieckiem a także skrócenie czasu potrzebnego na przygotowywanie dokumentów formalnych związanych z placówką oświatową. Szkoła jest miejscem w którym młodzi ludzie spędzają dużą część swojego czasu. Dotyczy to zarówno zajęć jak i czasu poświęconego na aktywności związane z przedmiotami szkolnymi poza jej murami. Uczniowie oprócz odrabiania lekcji muszą pamiętać o sprawdzianach, opłatach i wydarzeniach które organizuje szkoła. Comiesięczne zebrania z rodzicami nie pozwalają nauczycielom w wystarczającym stopniu przekazać informacji o postępach w kształceniu ich pociech. \\ \\
Wierzymy, że wraz z rozwojem platformy eSzkoła sytuacja ta ulegnie znacznej poprawie. Możliwość sprawdzenia ocen, zmian w planie lekcji oraz dostęp do materiałów udostępnionych przez prowadzących zajęcia przy użyciu komputera pozwoli na sprawniejsze kontrolowanie istotnych informacji. Elektroniczny system dzienniczka umożliwi nauczycielom natychmiastowy kontakt z rodzicami gdy zajdzie taka potrzeba. Moduł opłat przyspieszy dokonywanie przez rodziców przelewów a system kadr dostępny dla dyrektora i sekretariatu ułatwi kontrolowania wydatków placówki oświatowej. Możliwość rejestrowania wniosków o urlop, zwolnień lekarskich oraz podań usprawni przepływ dokumentów formalnych.

\newpage
\section{Opis zakresu projektu}
% wdrożenie w godz nocnych by nie zaburzyć pracy
% precyzyjnie określić czas działania
% obsłużenie żądań 100 użytkowników w czasie


Projekt składa się z następujących modułów
\begin{itemize}
    \item moduł dydaktyki
    \item moduł ocen
    \item moduł kont
    \item moduł administracji
\end{itemize}

Przedmiotem Umowy jest dostarczenie, wykonanie, instalacja i wdrożenie systemu \textit{eSzkoła}, obejmuje w szczególności:
%TODO dodać licencję
\begin{enumerate}
    \item wytworzenie, dostarczenie, zainstalowanie, skonfigurowanie i uruchomienie Systemu na zasobach sprzętowych wskazanych przez Zamawiającego, zgodnego z obowiązującym prawem;
    \item dostarczenie bazy danych zawierającej informacje dostarczone przez Zamawiającego w plikach zgodnych z formatem *.xls;
    \item skonfigurowanie i zintegrowanie dostarczanego Systemu do współpracy z systemami dzienniczka elektronicznego Librus i Vulcan a także zewnętrznego systemu kadr
\end{enumerate}

\subsection{Wymagania funkcjonalne}

\paragraph{Moduł dydaktyki} \mbox{}\\
\begin{enumerate}
	\item Materiały dydaktyczne to pliki dowolnego typu udostępniane w aplikacji. Nauczyciel ma możliwość pobierania, dodawania i usuwania ich, uczeń ma możliwość pobierania ich. Nauczyciel ma możliwość udostępniania materiałów na stronę przedmiotu, którego uczy. Jednorazowo wrzucone dane nie mogą przekraczać 10 MB.
    \item Plan zajęć klasy zawiera informacje o przedmiotach: godzinach ich rozpoczęcia i zakończenia, sali, w której się odbywają oraz nauczycielu prowadzącym zajęcia. Może być wyświetlany przez dowolny typ konta.
    \item Plan zajęć nauczyciela zawiera informacje o przedmiotach: godzinach ich rozpoczęcia i zakończenia, sali, w której się odbywają oraz klasie, z którą ma zajęcia. Może być wyświetlany przez dowolny typ konta.
    \item Kalendarium zawiera informacje o wydarzeniach szkolnych takich jak apele i uroczystości, a także dniach wolnych od zajęć dydaktycznych. Każde takie wydarzenie oprócz swojej daty posiada informację o godzinie rozpoczęcia i zakończenia, lokalizacji oraz krótką informację w postaci notki nie dłuższej niż 100 znaków.
    \item Kalendarz ucznia jest zbiorem spersonalizowany wydarzeń każdego ucznia. Zawiera informacje o zastępstwach nauczycieli, z którymi uczeń ma zajęcia, informacje o sprawdzianach (ich data oraz przedmiot) oraz o kołach naukowych, do których zapisany jest uczeń
    \item Klasa będąca zbiorem uczniów, posiada swoją nazwę oznaczaną jedną cyfrą (rok nauki) oraz jedną dużą literą alfabetu łacińskiego. Ma przypisanego wychowawcę (dokładnie jednego nauczyciela), oraz plan zajęć
    \item System oferuje integrację z istniejącymi rozwiązaniami dzienniczków elektronicznych LIBRUS oraz VULCAN na poziomie wyświetlania, modyfikacji oraz synchronizacji dwustronnej wprowadzonych ocen (tzn. wszelkie zmiany wprowadzone w dowolnym systemie powinny być odzwierciedlone w każdym z nich)
\end{enumerate}

\paragraph{Moduł ocen} \mbox{}\\
\begin{enumerate}
	\item Każde konto uprawnione do wystawiania ocen może to zrobić tylko w ramach swoich przedmiotów i tylko dla uczniów uczęszczających na te zajęcia.
    \item Każda ocena powinna być wartością całkowitoliczbową z zakresu 1-6 z ewentualnym modyfikatorem w postaci symbolu "-" lub "+" występującym raz na samym końcu wartości oceny, posiadać przypisaną wagę (domyślnie ustawioną jako 1), zawierać krótką informację o powodzie jej wystawienia nie dłuższą niż 60 znaków a także datę wystawienia.
    \item Oceny wprowadzone do systemu mogą być usuwane oraz modyfikowane. Modyfikacja może obejmować zmiany dowolnej wartości poza datą wystawienia oceny.
    \item Każde konto uprawnione do wystawiania ocen może wstawiać także punkty za aktywność lub nieprzygotowanie w postaci symboli "+" i "-". Takie punkty powinny zachować zasady wyświetlania i modyfikacji zgodnie z zasadami ocen.
	\item Każde trzy punkty "+" i każde trzy punkty "-" powinny mieć możliwość automatycznej zamiany na odpowiednio oceny 5 i 1 z przypisaną wagą oceny 1.
    \item Możliwe jest wyliczenie średniej ważonej na podstawie zgromadzonych przez ucznia ocen i wyświetlenie jej jako proponowanej oceny końcowej.
    \item Konto nauczyciela posiada możliwość zaakceptowania proponowanej oceny końcowej wyliczonej przez system jak również wprowadzenie własnej wartości.
    \item Konta rodzica oraz ucznia mogą obejrzeć wystawione oceny a także wszystkie związane z nimi informacje.
\end{enumerate}

\paragraph{Moduł kont} \mbox{}\\
\begin{enumerate}
	\item System wspiera obsługę powiadomień - specjalnych akcji które informują o ważnym dla użytkownika wydarzeniu. \item Mogą one obejmować takie sytuacje jak:
	\subitem wystawienie oceny
	\subitem pojawienie się nowej ankiety
	\subitem organizowyane zebrania
	\subitem uroczystości szkolne
	\subitem przyznane stypendia
	\subitem otrzymane świadectwa
	\subitem nadchodzące egzaminy
	\item Powiadomienia zawierające informacje w formie tekstowej nie dłuższej niż 1024 znaków wysyłane są w formie mailowej lub SMS zgodnie z preferencjami zaznaczonymi na koncie użytkownika
	\item System wspiera obsługę ankiet. Są to specjalne dokumenty zawierające listę pytań z powiązaną im listą odpowiedzi. Udzielane odpowiedzi mogą być jednokrotnego lub wielokrotnego wyboru

\end{enumerate}

\textbf{Konto rodzica}
\begin{enumerate}
	\item Konto zawiera dane profilowe będące informacjami o: imieniu, nazwisku, numerze telefonu, adresie email, adresie do korespondencji, numerze konta bankowego.
	\item Użytkownik ma możliwość podglądu danych profilowych, oraz ich modyfikacji.
	\item Użytkownik ma dostęp do rozszerzonego zestawu powiadomień, który poza standardowym zakresem akcji obejmuje również:
	\begin{enumerate}
		\item nieobecności ucznia (z możliwością filtrowania: na zajęciach, pracach klasowych, zajęciach pozalekcyjnych)
		\item spóźnienia ucznia
		\item pojawienie się nowych opłat
		\item zbliżające się terminy opłat (z możliwością konfiguracji liczby dni przed terminem opłaty, dla której ma pojawić się powiadomienie)
	\end{enumerate}

	\item Użytkownik w przypadku gdy jest opiekunem więcej niż jednego Ucznia ma możliwość wyboru konta Ucznia w ramach, którego przegląda system
	\item Użytkownik ma możliwość podglądu osiągnięć Ucznia oraz uwag dotyczących jego sprawowania. Rodzic ma dostęp do informacji o ocenach, nieobecnościach i spóźnieniach Ucznia.
	\item Użytkownik może wyświetlić informacje o szczególnych osiągnięciach Ucznia:
	\begin{enumerate}
		\item wygranych w konkursach naukowych
		\item wygranych w zawodach sportowych
		\item osiągnięciach artystycznych
	\end{enumerate}
	\item Użytkownik ma możliwość podglądu planu zajęć Ucznia wraz z naniesionymi na niego aktualnymi zmianami (uroczystości szkolne, nieobecności nauczycieli)
	\item Użytkownik ma dostęp do informacji o terminach planowanych prac klasowych Ucznia oraz o obowiązującym zakresie materiału
	\item System umożliwi Użytkownikowi usprawiedliwianie oraz zwalnianie Ucznia z zajęć lekcyjnych
	\item Użytkownik będzie miał pełną informację o samorządzie klasowym Ucznia, o pełnionych przez uczniów funkcjach i zakresie ich obowiązków
	\item Użytkownik będzie miał wgląd do wydarzeń organizowanych w szkole wliczając w to uroczystości szkolne, organizowane wycieczki, wydarzenia sportowe na terenie i poza szkołą
	\item Użytkownik może wyświetlić informacje na temat godzin dyżurów nauczycieli, w trakcie których Rodzic będzie mógł porozmawiać na temat swojego podopiecznego
	\item Użytkownik ma możliwość uregulowania opłat za pośrednictwem systemu. Funkcjonalność ta będzie wprowadzona poprzez integrację z systemem płatności elektronicznych za pomocą przelewów PayU, kart kredytowych/debetowych eService oraz systemem PayPal
	\item Użytkownik ma możliwość wygenerowania dowodów wniesionych opłat w postaci plików PDF
	\item Użytkownik ma dostęp do listy stypendiów przyznanych Uczniowi. Stypendia będą wpłacane na konto bankowe, które podał Rodzic podczas konfiguracji swojego konta użytkownika
	\item Użytkownik ma dostęp do listy wszystkich Świadectw, które otrzymał Uczeń w procesie kształcenia
	\item Użytkownik ma dostęp do informacji o kołach naukowych (zainteresowań) do których należy Uczeń. Rodzic ma również podgląd listy płatnych zajęć pozalekcyjnych odbywających się na terenie szkoły z możliwością zapisania Ucznia na takie zajęcia.
	\item Zebrania rodziców zwoływane przez nauczycieli lub dyrekcję będą wyświetlane w panelu Rodzica. Użytkownik będzie miał możliwość potwierdzenia swojej obecności, lub poinformowaniu o swojej absencji
	\item Katalog przedmiotów obowiązkowych w przebiegu edukacji szkolnej dla danego ucznia będzie udostępniony Użytkownikowi. Katalog będzie zawierał informacje na temat poszczególnych przedmiotów w zakresie liczby godzin lekcyjnych, zakresu materiału oraz realizowanych wymaganiach egzaminacyjnych w ramach przedmiotu
	\item Użytkownik ma dostęp w postaci hiperłączy do informatorów udostępnianych przez CKE dotyczących egzaminów zdawanych na zakończenie szkoły
	\item Użytkownik będzie miał możliwość uzupełniania ankiet przygotowanych przez administrację szkolną oraz podglądu wyłącznie udostępnionych przez administrację wyników
\end{enumerate}
	Dodatkowo Rodzic może mieć rozszerzoną funkcjonalność konta w przypadku gdy jest przedstawicielem Komitetu Rodzicielskiego.
	\begin{enumerate}
		\item Użytkownik ma możliwość organizowania zbiórki pieniężnej na cele Komitetu Rodzicielskiego
		\item Użytkownik ma dostęp do panelu umożliwiającego tworzenie ankiet skierowanych do pozostałych Rodziców w klasie Ucznia
		\item Użytkownik ma możliwość zwoływania zebrań Komitetu Rodzicielskiego
	\end{enumerate}


\textbf{Konto ucznia}
\begin{enumerate}
	\item Konto zawiera dane profilowe będące informacjami o: imieniu, nazwisku, numerze telefonu, adresie email, adresie do korespondencji
	\item Użytkownik ma możliwość obejrzenia danych profilowych, oraz ich modyfikację
	\item Użytkownik może zmienić sposób otrzymywania powiadomień
	\item Z kontem związane mogą być informacje zwane osiągnięciami, będące tekstem nie dłuższym niż 256 znaków opisującym wygrane w konkursach naukowych, zawodach sportowych lub osiągnięcia artystyczne
	\item Użytkownik może wyświetlić listę swoich osiągnięć
	\item Użytkownik może wyświetlić obowiązujący plan zajęć wraz z naniesionymi poprawkami lub tymczasowymi zmianami
	\item Użytkownik może wyświetlić informacje o terminach prac klasowych wraz z zakresem obowiązującego materiału
	\item Użytkownik może obejrzeć informacje o samorządzie klasowym - liście osób wraz z pełnionymi przez nie funkcjami
	\item Użytkownik ma dostęp do tabeli z wypisanymi godzinami dyżurów nauczycieli
	\item Użytkownik może wyświetlić listę przyznanych my stypendiów oraz wystawionych świadectw
	\item Użytkownik może wyświetlić informacje dotyczące przebiegu edukacji szkolnej zawierającej takie informacje jak liczba godzin lekcyjnych, zakres materiału lub realizowane wymagania egzaminacyjne
	\item Użytkownik ma dostęp w postaci hiperłączy do informatorów udostępnianych przez CKE dotyczących egzaminów zdawanych na zakończenie szkoły
	\item Użytkownik ma możliwość uzupełniania ankiet wysyłanych przez administrację szkolną

\end{enumerate}

\textbf{Konto nauczyciela}
\begin{enumerate}
  \item Konto zawiera dane profilowe będące informacjami o: imieniu, nazwisku, numerze telefonu, adresie email, adresie do korespondencji
  \item Użytkownik ma możliwość obejrzenia danych profilowych, oraz ich modyfikację
  \item Konto nauczyciela umożliwia mu pracę w trzech trybach, których wykorzystanie uzależnione będzie od aktualnie pełnionej funkcji. Tryby pracy konta nauczyciela to: tryb prowadzącego lekcję, tryb zarządzania przedmiotem, tryb wychowawcy
  \item Tryb prowadzącego lekcję będzie umożliwiał nauczycielowi wykorzystanie funkcjonalności systemu związanych z prowadzeniem zajęć lekcyjnych. Użytkownik będzie przechodzić do trybu prowadzącego lekcję po rozpoczęciu wcześniej umieszczonej w planie lekcji
  \item Po zainicjowaniu trybu lekcji nauczyciel ponaglony będzie do uzupełnienia obecności uczniów danej klasy oraz do ewentualnej edycji i zaakceptowania tematu lekcji. Po sprawdzeniu obecności nauczyciel będzie mógł edytować stan obecności zmieniając nieobecność ucznia na spóźnienie
  \item W trybie lekcji nauczyciel przyznawać może uczniom oceny oraz dodawać uwagi dotyczące ich zachowania
  \item Nauczyciel może przypisać do lekcji zadanie domowe. Termin jego oddania ustawiany będzie przez nauczyciela - zadanie będzie przypisane do jednej z przyszłych lekcji danego przedmiotu. Informacje o zadaniach domowych przypisanych do danej lekcji wyświetlane będą nauczycielowi po rozpoczęciu lekcji. W każdym momencie będzie miał on możliwość edycji treści zadania domowego.
  \item Zadanie domowe będzie posiadać tytuł oraz opis słowny. Dodatkowo nauczyciel będzie mógł dodać załącznik w dowolnym formacie będący uzupełnieniem treści zadania
  \item W trybie zarządzania przedmiotem nauczyciel będzie miał możliwość uzupełnienia przedmiotu o zakres materiału, rozplanowaniu tematów na poszczególne lekcje, informacje o realizowanym zakresie programowym, obowiązujące podręczniki oraz literaturę uzupełniającą
  \item W trybie zarządzania przedmiotem nauczyciel będzie mógł zaplanować prace pisemne (sprawdziany, prace klasowe)
  \item Tryb zarządzania przedmiotem pozwoli nauczycielowi na zebranie statystyk dotyczących przedmiotu dla każdej z klas, w której uczy danego przedmiotu. będzie miał on dane statystyczne ocen ze sprawdzianów, ocen semestralnych oraz rocznych.
  \item Nauczyciel będzie miał możliwość eksportu powyższych danych do arkusza kalkulacyjnego programu Microsoft Office Excel
  \item Tryb wychowawcy będzie odblokowany dla nauczycieli będącymi wychowawcami klas
  \item Dla klasy, której użytkownik jest wychowawcą, będzie mógł on dodawać informacje wydarzeniach szkolnych,wycieczkach szkolnych
  \item Nauczyciel będący wychowawcą klasy będzie mógł zwoływać zebrania rodziców
  \item Wychowawca będzie miał możliwość dodawania zbiórek pieniężnych
  \item Wychowawca klasy będzie miał uprawnienia nadawania kontom rodziców uprawnień członków komitetu rodzicielskiego oraz nadawania kontom uczniów statusu członka samorządu klasowego oraz przypisywanie im funkcji przewodniczącego, skarbnika, sekretarza oraz członka
  \item Wychowawca klasy będzie miał możliwość zebrania danych dotyczących absencji uczniów, spóźnień, uwag dotyczących zachowania oraz ich wyeksportowanie do arkusza kalkulacyjnego programu Microsoft Office Excel
  \item Wychowawca klasy będzie miał możliwość przeprowadzenia ankiet wśród uczniów lub rodziców uczniów
  \item Wychowawca będzie miał możliwość usprawiedliwienia nieobecności ucznia na zajęciach lub jego zwolnienie
  \item Wychowawca będzie miał możliwość przypisania uczniom semestralnej oraz rocznej oceny zachowania
  \item Ponadto nauczyciel będzie miał dostęp do funkcjonalności niezwiązane z żadnym z trzech trybów. Należeć będą do nich dodawanie do planu terminu swoich dyżurów, wprowadzenie informacji o własnych nieobecnościach
\end{enumerate}

\paragraph{Moduł administracji} \mbox{}\\
Moduł ten służy zarządzaniu użytkownikami systemu a także spełnia rolę systemu kard, który umożliwia prowadzenie ewidencji pracowników szkoły.\\Dostęp do modułu mają jedynie wybrane osoby z administracji placówki oświatowej posiadające status administratora systemu. Konto administratora ma dostęp do następujących czynności i informacji oraz składa się z elementów wyszczególnionych w poniższym opisie:
\begin{enumerate}
\item W systemie administracji znajduje się lista wszystkich użytkowników, których administrator może swobodnie wyszukiwać i filtrować.
\item System pozwala na dodanie nowego konta użytkownika do systemu: ucznia, nauczyciela lub rodzica. Przy tej operacji niezbędne jest podanie podstawowych danych ucznia, które będą dostępne w profilu ucznia.
\item Konto administratora umożliwia edycję danych personalnych dowolnego użytkownika systemu.
\item Istnieje możliwość usunięcia dowolnego użytkownika systemu.
\item System oferuje możliwość edycji ocen wystawionych przez nauczyciela.

\item Administrator może zmienić status pracy nauczyciela: aktywny, urlopowany, niepracujący, zwolniony, emerytowany, na zwolnieniu lekarskim.
\item Administrator jest odpowiedzialny za rozpatrywanie wniosków urlopowych nauczycieli. Może wyrazić zgodę na urlop lub odrzucić wniosek. Wtedy Nauczyciel otrzymuje stosowny komunikat o zmianie statusu wniosku.
\item Administrator ma pełny dostęp do edycji kalendarza szkolnego, klas, kół naukowych.
\item System oferuje pomoc przy wyborze zastępstwa za nauczyciela:
W przypadku, gdy nauczyciel przechodzi na urlop, system proponuje administratorowi możliwe zastępstwa godzin lekcyjnych nauczyciela przez innych ,,wolnych'' w tym momencie nauczycieli.
\item Administrator ma dostęp do danych nauczycieli, również tych niedostępnych z innych części systemu:
\begin{itemize}
\item szczegółowe dane personalne,
\item dane kontaktowe,
\item aktualne wynagrodzenie,
\item stanowisko,
\item staż pracy,
\item typ umowy,
\item ukończone kursy i szkolenia,
\item badania lekarskie,
\item wykorzystany urlop.
\end{itemize}
\item System oferuje możliwość stworzenia nowego planu zajęć oraz modyfikacje istniejącego planu. Plan można wyeksportować do formatu .xls.

\end{enumerate}


% dostarczenie bazy danych zawierającej informacje o planie zajęć, kołach naukowych oraz uroczystościach szkolnych dostarczonych przez Zamawiającego w formie elektronicznej zgodnej z formatem pliku *.xls;
\subsection{Wymagania niefunkcjonalne}
\begin{enumerate}
	\item System powinien obsługiwać do 5000 użytkowników wszystkich typów kont.
	\item System powinien obsłużyć do 1000 jednoczesnych użytkowników bez zaistnienia opóźnień większych niż:
	\begin{enumerate}
		\item 	do 10 użytkowników 3s
		\item do 100 użytkowników 5s
		\item do 1000 użytkowników 10s
	\end{enumerate}
	\item W przypadku większej liczby użytkowników podłączonych równocześnie system powinien zachować stabilność lecz może zwiększyć opóźnienia
	\item Aplikacja webowa będzie napisana z wykorzystaniem nowoczesnych technologii HTML5, CSS3, JS i będzie kompatybilna z przeglądarkami:
	\begin{enumerate}
		\item Google Chrome $48>$
		\item Firefox $44>$
		\item Safari $9>$
		\item Internet Explorer $10>$
	\end{enumerate}
	\item Aplikacja powinna zostać zaimplementowana w technologiach, które można uruchomić na Serwerze systemem Windows Server 2012
	\item Szata graficzna powinna być podobna do systemu USOS
	\item System powinien spełniać standardy WCAG2.0
	\item System powinien być zintegrowany z systemem PayU, PayPal, eService w celu dokonywania płatności
	\item System powinien automatycznie wykonywać backup’y bazy danych codziennie pomiędzy godziną 2:00-5:00
	\item System powinien być tworzony zgodnie ze standardami przyrostowego sposobu  wytwarzania oprogramowania
	\item System powinien być prosty i intuicyjny w użyciu
	\item Maksymalne wymagania systemowe Systemu:
	\begin{enumerate}
		\item 64GB RAM
		\item 1TB pamięci dyskowej SSD
		\item moc obliczeniowa 10xCPU Intel Penitum II Xeon
	\end{enumerate}


\end{enumerate}


\subsection{Wymagania w zakresie gwarancji Systemu}
\begin{enumerate}
	\item Gwarancja świadczona przez Wykonawcę będzie dotyczyć Systemu, który opisany jest w niniejszym dokumencie.
	\item Wykonawca zobowiązuję się do przyjmowania zgłoszeń o awarii przez 7 dni w tygodniu, 24h na dobę.
	\item Czas reakcji serwisowej wynosi 4 godziny od otrzymania zgłoszenia od Zamawiającego.
	\item Wykonawca zobowiązuję się usunąć awarię w następującym czasie od momentu zgłoszenia przez Zamawiającego:
	\begin{enumerate}
		\item dla błędów krytycznych - \textbf{12h}
		\item dla błędów ważnych - \textbf{48h}
		\item dla błędów zwykłych - \textbf{7dni}
	\end{enumerate}

	\item Priorytet błędu ustalany jest przez zamawiającego w dokumencie zgłoszenia o wystąpieniu błędu, zgodnie z definicjami:
	\begin{enumerate}
		\item \textbf{błąd krytyczny} - powoduje zatrzymanie pracy Systemu, bądź ma krytyczny wpływ na jego funkcjonalność,
		\item \textbf{błąd ważny} - powoduje poważne utrudnienia w pracy Systemu; powtarzające się zakłócenia
		\item \textbf{błąd zwykły} - problem, który nie powoduje bezpośredniego wpływu na funkcjonalność systemu.
	\end{enumerate}

	\item Zaplanowane przerwy działania systemu mogą mieć miejsce w dni wolne od pracy lub w dni robocze w godzinach 1:00-6:00. Każda planowana przerwa w systemie powinna zostać zgłoszona Zamawiającemu z wyprzedzeniem 4 dni.

\end{enumerate}
\subsection{Sposób wdrożenia}
% Łukasz
Wymagane jest, aby Wykonawca zarządzał procesem wdrożenia tak, aby możliwe było nadzorowanie nie tylko rezultatów wdrożenia, ale również całego procesu.
\paragraph{Etapy}
\begin{enumerate}
\item \textbf{Etap Planowania.} 
Na Etapie Planowania Wykonawca, wykorzystując przekazane wymagania dla Systemu, opracuje Projekt Wykonawczy Wdrożenia Systemu, Dokumentację Testową oraz Dokumentację Zarządczą. Opracowane na tym etapie dokumenty muszą opisywać oba środowiska: testowe i produkcyjne. Kompletność i spójność dokumentów Etapu Planowania będzie następnie weryfikowana podczas kolejnych etapów procesu wdrożenia.
Produkty etapu:
\begin{enumerate}
\item \textbf{Projekt Wykonawczy Wdrożenia Systemu.}
Dokument będzie opisywał szczegółowo architekturę systemu, sposób konfiguracji i funkcje poszczególnych elementów systemu oraz sposób realizacji wymagań przez system. W szczególności projekt będzie zawierał specyfikacje oprogramowania i licencji, które zostaną dostarczone podczas Etapu Dostawy Oprogramowania i Licencji. 
\item \textbf{Plan i Scenariusze Testów Akceptacyjnych Wdrożenia Systemu.} 
Element Dokumentacji Testowej. Wykonawca opracuje Plan i Scenariusze Testów Akceptacyjnych Wdrożenia Systemu. Zakres opracowanych scenariuszy musi umożliwić jednoznaczną weryfikację spełnienia wymagań dla Systemu określonych przez Zamawiającego.
\item \textbf{Harmonogram Wdrożenia Systemu.}
Element Dokumentacji Zarządczej. Harmonogram zostanie opracowany przez Wykonawcę. W szczególności harmonogram musi uwzględniać harmonogram i powiązania wynikające z wdrożenia Systemu.
\item \textbf{Macierz Odpowiedzialności.}
Element Dokumentacji Zarządczej. Wykonawca opracuje Macierz Odpowiedzialności, której celem jest przypisanie odpowiedzialności za zidentyfikowane obszary i zadania wykonywane w ramach wdrożenia do konkretnych podmiotów i osób. Obszary i zadania muszą zostać zdefiniowane do takiego poziomu szczegółowości, aby przypisanie strony za nie odpowiedzialnej było jednoznaczne.
\end{enumerate}
\item \textbf{Etap Dostawy Oprogramowania i Licencji.}
Zakres prac Wykonawcy obejmie dostawę niezbędnych licencji wraz z nośnikami oprogramowania wskazanego w Projekcie Wykonawczym Wdrożenia Systemu. Dopuszcza się dostarczenie licencji i oprogramowania w kilku kolejnych transzach, jeżeli jest to zgodne z Harmonogramem Wdrożenia Systemu. Produkty etapu:
\begin{enumerate}
\item Licencje na oprogramowanie wraz z dokumentacją producenta.
\item Dostarczone nośniki z oprogramowaniem.
\item Podpisany Protokół Odbioru Ilościowego Oprogramowania zawarty w Załączniku nr 6.
\end{enumerate}
\item \textbf{Etap Instalacji Oprogramowania i Wdrożenia Systemu.}
Zakres prac Wykonawcy obejmuje instalację i konfigurację oprogramowania systemowego oraz oprogramowania Systemu, przeprowadzenie testów odbiorowych zgodnie z Planem Testów Akceptacyjnych Wdrożenia Systemu. Przedstawiony zakres musi być zrealizowany w obu środowiskach: testowym i produkcyjnym. Przygotowanie niezbędnej infrastruktury sprzętowej, w szczególności: montaż fizyczny, podłączenie do instalacji zasilającej oraz wykonanie niezbędnych połączeń sieci LAN leży po stronie Zamawiającego. Wszystkie działania w ramach etapu powinny zostać przeprowadzone w terminach zgodnych z Harmonogramem Wdrożenia Systemu i zakończone pozytywnym wynikiem Testów Akceptacyjnych Wdrożenia Systemu. Produkty etapu:
\begin{enumerate}
\item Wdrożona funkcjonalność Systemu.
\item Protokoły Testów Akceptacyjnych Wdrożenia Systemu.
\end{enumerate}
\item \textbf{Etap Dokumentacji.} Zakres prac Wykonawcy w trakcie etapu obejmuje przygotowanie dokumentacji w postaci:
\begin{enumerate}
\item Procedur operacyjnych Systemu;
\item Procedur administracyjnych Systemu
\end{enumerate}
Opracowane na tym etapie dokumenty muszą opisywać oba środowiska: testowe i produkcyjne.
\item \textbf{Etap Przekazania do Eksploatacji.} Zakres prac Wykonawcy w trakcie etapu obejmuje przekazanie Zamawiającemu wszystkich haseł i praw administracyjnych oraz udzielanie przez okres 2 tygodni wsparcia w rozwiązywaniu problemów powstających podczas wykonywania czynności administracyjnych. Wykonawca będzie świadczył wsparcie telefonicznie lub, jeżeli problem tego wymaga, w siedzibie Zamawiającego. Przedstawiony zakres dotyczy obu środowisk: testowego i produkcyjnego. Produkty etapu:
\begin{enumerate}
\item Podpisany przez Zamawiającego i Wykonawcę Protokół Przekazania do Eksploatacji;
\item Raport z Okresu Wsparcia  – przygotowany przez Wykonawcę;
\end{enumerate}
\end{enumerate}
Warunkiem zakończenia realizacji kolejnych etapów jest zaakceptowanie przez Zamawiającego wszystkich uzgodnionych produktów oraz podpisanie przez Zamawiającego Protokołów Odbioru.
\paragraph{Przebieg szkoleń}
W ramach wdrożenia systemu odbędzie się cykl szkoleń przeznaczonych dla uczniów, rodziców, nauczycieli i administratorów. Szkolenia będą prowadzone w salach szkolnych przeznaczonych do nauki informatyki - wyposażone w stanowiska komputerowe, projektor oraz tablicę do rysowania.
\begin{itemize}
\item Szkolenia dla uczniów odbywają się w ramach przedmiotu ,,informatyka''.
\item Szkolenia dla chętnych rodziców odbywają się raz na grupę osób w godzinach 17:00 - 19:00.
\item Szkolenia dla pracowników szkoły odbywają się w godzinach 09:00-12:00.
\end{itemize}
Pracownicy oddelegowani do udziału w szkoleniach obowiązani są uczestniczyć w nich w pełnym wymiarze czasu. Po zakończeniu szkoleń wszyscy uczestnicy wypełniają ,,Ankietę oceny szkolenia''.

Do systemu dołączony zostanie ,,samouczek'' - dokument opisujący działanie poszczególnych części systemu.

\subsection{Warunki odbioru systemu}
Kryterium akceptacji wdrożenia jest pozytywny wynik testów akceptacyjnych (w obu środowiskach: testowym i produkcyjnym) określonych w Dokumentacji Testowej oraz akceptacja przez Zamawiającego wszystkich produktów poszczególnych Etapów. 

\subsection{Procedury odbioru systemu}
\begin{enumerate}
\item \textbf{Procedura Odbioru Dokumentacji}
\begin{enumerate}
\item Poniższa procedura ma zastosowanie do wszystkich dokumentów powstałych w wyniku wykonywania przedmiotu Umowy.
\item W terminie przewidzianym w Harmonogramie Wykonawca przekazuje Zamawiającemu, za pokwitowaniem, dokument przeznaczony do odbioru.
\item W terminie 5 dni roboczych od przekazania dokumentu Zamawiający przekazuje Wykonawcy podpisany przez Zamawiającego Protokół Odbioru Dokumentu, w którym Zamawiający:
\begin{enumerate}
\item odbiera dokument bez zastrzeżeń,
\item odbiera dokument z uwagami,
\item odrzuca dokument w całości.
\end{enumerate}
\item Wzór Protokołu Odbioru Dokumentu został zawarty w Załączniku nr 4 do Umowy.
\item W przypadku odebrania dokumentu z uwagami Zamawiający dołącza do Protokołu Odbioru Dokumentu wykaz uwag, a w przypadku odrzucenia dokumentu w całości – pisemne uzasadnienie decyzji z przytoczeniem jej powodu bądź powodów.
\item W przypadku odbioru dokumentu z uwagami lub odrzucenia dokumentu w całości Wykonawca, w terminie 5 dni roboczych od daty przekazania Protokołu Odbioru Dokumentu przez Zamawiającego, jest zobowiązany do poprawienia dokumentu i ponownego przedstawienia go do odbioru.
\item Jeżeli w terminie 5 dni roboczych od daty ponownego przekazania dokumentu do odbioru,  Zamawiający ponownie odbierze dokument z uwagami lub odrzuci dokument.
\end{enumerate}
\item \textbf{Procedura Odbioru Ilościowego Oprogramowania}
\begin{enumerate}
\item Strony ustalają, że dostawy Oprogramowania (Dostawy) będą odbywać się w dni robocze, zgodnie z Harmonogramem, przy czym Wykonawca, na 5 dni roboczych przed terminem każdej Dostawy, zobowiązany jest uprzedzić Zamawiającego o planowanej Dostawie, wskazując dodatkowo przewidywaną godzinę Dostawy oraz dane osób realizujących Dostawę.
\item W ciągu 2 dni roboczych od dnia Dostawy i w miejscu Dostawy Zamawiający dokona, przy udziale przedstawiciela Wykonawcy, odbioru ilościowego dostarczonego Oprogramowania.
\item Odbiór ilościowy będzie polegał na sprawdzeniu ilościowym elementów Dostawy, sprawdzeniu kompletności i stwierdzeniu braków, uszkodzeń mechanicznych, a także sprawdzeniu zgodności Dostawy z terminem realizacji Umowy wskazanym w Harmonogramie, co zostanie potwierdzone Protokołem Odbioru Ilościowego Oprogramowania.
\item Zamawiający przekaże Wykonawcy podpisany Protokół Odbioru Ilościowego Oprogramowania, w którym Zamawiający odbiera Dostawę bez zastrzeżeń, odbiera Dostawę z uwagami lub też odrzuca Dostawę w całości. W przypadku odbioru Dostawy z zastrzeżeniami Zamawiający dołącza do Protokołu Odbioru Ilościowego Oprogramowania wykaz uwag, a w przypadku odrzucenia Dostawy w całości – pisemne uzasadnienie wskazujące powody odrzucenia Dostawy.
\item Protokół Odbioru Ilościowego Oprogramowania musi potwierdzać Dostawę do właściwego miejsca oraz zawierać ilości i dane identyfikacyjne Oprogramowania.
\item W przypadku odrzucenia Dostawy w całości przez Zamawiającego Wykonawca, w terminie uzgodnionym z Zamawiającym, zobowiązany jest dokonać ponownej Dostawy Oprogramowania z tym, że przy ponownym dokonywaniu Dostawy Wykonawca jest w opóźnieniu.   
\item W przypadku dokonania przez Zamawiającego odbioru ilościowego Oprogramowania z uwagami Wykonawca zobowiązany jest w ciągu 10 dni roboczych od Dostawy dokonać Dostawy uzupełniającej brakującego lub uszkodzonego Oprogramowania. Jeżeli w tym terminie Wykonawca, w ramach Dostawy uzupełniającej, dostarczy Zamawiającemu brakującą część Oprogramowania objętego daną Dostawą i zostanie ona odebrana bez zastrzeżeń, Dostawę uznaje się za dokonaną w terminie Dostawy.
\end{enumerate}
\item \textbf{Procedura Odbioru Testów Akceptacyjnych}
\begin{enumerate}
\item Niezwłocznie po zakończeniu prac nad wdrożeniem Systemu Wykonawca powiadomi Zamawiającego o gotowości do przeprowadzenia Testów Akceptacyjnych Systemu.
\item Testy Akceptacyjne Systemu przeprowadzane będą przez Zamawiającego z udziałem Wykonawcy, w terminie wskazanym w Harmonogramie.
\item Pozytywne zakończenie Testów Akceptacyjnych, oznaczające pozytywny rezultat dla wszystkich przypadków testowych zdefiniowanych w zaakceptowanych scenariuszach Testów Akceptacyjnych, potwierdzone zostanie podpisaniem Protokołu Testów Akceptacyjnych.
\item W przypadku negatywnego rezultatu Testów Akceptacyjnych Wykonawca, w terminie uzgodnionym z Zamawiającym, usunie wady i ponownie zgłosi gotowość do przeprowadzenia Testów Akceptacyjnych. 
\item W przypadku ponownego negatywnego rezultatu Testów Akceptacyjnych Zamawiający wyznaczy termin kolejnego przeprowadzenia Testów Akceptacyjnych Systemu nie dłuższy niż 7 (siedem) dni roboczych. W przypadku upływu wyznaczonego terminu i nie wywiązania się Wykonawcy ze wskazanych uchybień w całości, Zamawiający uprawniony będzie do wyznaczenia kolejnego terminu albo odstąpienia od Umowy w całości lub w części, według własnego wyboru, w terminie 30 dni od dnia upływu wyznaczonego terminu. 
\end{enumerate}
\item \textbf{Procedura Odbioru Szkolenia}
\begin{enumerate}
\item Po zakończeniu każdego szkolenia Wykonawca i Zamawiający podpiszą Protokół Odbioru Szkolenia.
\item Warunkiem podpisania przez Zamawiającego Protokołu Odbioru Szkolenia jest załączenie przez Wykonawcę do protokołu listy obecności uczestników szkolenia i kopii imiennych certyfikatów ukończenia szkolenia wszystkich uczestników.
\end{enumerate}
\end{enumerate}

\section{Utrzymanie systemu po wdrożeniu}
odporność na błędy, że działa przez jakiś okres czasu
%TODO warunki co do ludzi utrzymujących: wykształcenie, certyfikaty

\section{Dokumentacja systemu}
% Pzypadkiem zrobiliśmy też to
\paragraph{Dokumentacja użytkownika} komplet dostarczonej dokumentacji będzie zawierać podręczniki dla użytkowników systemu zgodnie ze zdefiniowanymi w Systemie rolami. Podręcznik będzie zawierał wykaz czynności wykonywanych przez użytkownika pełniącego ustaloną rolę oraz szczegółowy sposób realizacji tych czynności (kolejne kroki) wraz ze zrzutami ekranów. Dokumentacja zostanie dostarczona w formie elektronicznej umożliwiającej wydruk.
\paragraph{Dokumentacja administratora}  w skład dokumentacji technicznej administratora wejdą dokumenty dotyczące następujących zagadnień:
\begin{enumerate}
	\item użyte w projekcie oprogramowanie systemowe i narzędziowe, ze wskazaniem wersji oraz sposobu konfiguracji
	\item lista wykorzystanych bibliotek wraz ze wskazaniem wersji;
	\item sposób instalacji i konfiguracji wszystkich składników sprzętu i oprogramowania;
	\item dokumentacja struktur baz danych oraz konfiguracji poszczególnych elementów: serwerów, urządzeń sieciowych, aplikacji.
	\item procedury tworzenia kopii i odtwarzania poszczególnych elementów Systemu.
\end{enumerate}
\paragraph{Testy akceptacyjne} zostaną opracowane przez Wykonawcę. Plan i scenariusze testów będą zgodne z powszechnie stosowanymu zasadami i praktykami. Przygotowany plan testów będzie określał:
\begin{enumerate}
	\item zasady przeprowadzania testów
	\item kolejność wykonywania scenariuszy testowych
	\item kryteria sukcesu dla poszczególnych scenariuszy
\end{enumerate}
Scenariusze będą pokrywały wszystkie wyspecyfikowane funkcje systemu. Każdy scenariusz określać będzie: dane, które muszą być wprowadzone do Systemu przed uruchomieniem scenariusza; kolejność czynności, wykonywanych w czasie testu oraz dane, wprowadzane do Systemu w czasie testu; oczekiwaną reakcję Systemu na wykonane czynności i wprowadzone dane. Dane testowe (do przeprowadzenia testów akceptacyjnych) w tym wszelkie materiały eksploatacyjne dostarczone będą przez Wykonawcę.

\section{Zasoby}
Oświadczamy, iż dysponujemy następującymi zasobami:
\begin{itemize}
    \item Pliki w formacie *.xls zawierające wszystkie plany zajęć klas oraz plany zajęć nauczycieli
    \item Pliki w formacie *.xls zawierające spis uczniów z informacjami: imię, nazwisko, data urodzenia, pesel, adres, data rozpoczęcia szkoły, klasa, do której uczeń jest zapisany
    \item Pliki w formacie *.xls zawierające spis opiekunów uczniów z informacjami: imię i nazwisko opiekuna, pesel ucznia, adres, telefon kontaktowy
    \item Pliki w formacie *.xls zawierające spis nauczycieli z informacjami: imię, nazwisko, adres, telefon kontaktowy, konto bankowe %TODO czy uwzględniamy wypłaty nauczycieli?
\end{itemize}

Wykonujący musi dysponować następującymi zasobami osobowymi:
\begin{itemize}
    \item co najmniej dwóch specjalistów baz danych o nie mniej niż dwuletnim doświadczeniu na podobnym stanowisku
    \item co najmniej czterech programistów posiadających certyfikat potwierdzający ich umiejętności programowania w języku Fortran
\end{itemize}


\section{Harmonogram wykonania Umowy}
\begin{enumerate}
    \item Umowa będzie realizowana od daty jej zawarcia i w okresie nie dłuższym niż do dnia 30 czerwca 2018 r., zgodnie ze szczegółowym Harmonogramem
    \item Wykonawca w terminie do 10 dni od dnia zawarcia Umowy, wytworzy i dostarczy do akceptacji Zamawiającego Harmonogram uwzględniający terminy i etapy zawarte w Umowie. Wykonawca uwzględni uwagi Zamawiającego dotyczące opracowanego Harmonogramu prac
    \item Wykonawca jest zobowiązany do realizacji przedmiotu Umowy w sposób następujący:
    \begin{enumerate}
        \item etap 1 - analiza przedwdrożeniowa - opracowanie projektu technicznego systemu eSzkoła w terminie do 30 dni od dnia zawarcia Umowy
        \item etap 2 - implementacja systemu w terminie do 275 dni od dnia zawarcia      Umowy nie później jednak niż do dnia 20 czerwca 2017 r.
        \item etap 3 - integracja systemu eSzkoła z istniejącymi rozwiązaniami dzienniczków elektronicznych LIBRUS oraz VULCAN w terminie nie późniejszym niż do dnia 20 czerwca 2017 r.
        \item Po realizacji etapów 1, 2 i 3 Wykonawca zobowiązuje się do wsparcia technicznego oraz zapewnienia usług serwisowych przez okres roku nie później niż do 30 czerwca 2018 r.
    \end{enumerate}
\end{enumerate}

\section{Kary umowne}
\end{document}
