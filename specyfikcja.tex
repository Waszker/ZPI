\documentclass{article}
\usepackage[MeX]{polski}
\usepackage[utf8]{inputenc}
\author{Fundacja nowoczesna szkoła\\\\
Imiona i nazwiska\\członków zespołu} % TODO
\title{eSzkoła - system zarządzania placówką oświatową}

\begin{document}
\maketitle
\newpage
\tableofcontents
\newpage

\section{Opis projektu}
% TODO napisać opis projektu

\section{Cele projektu}
Głównym celem zamawianego produktu jest usprawienie działania placówki oświatowej pod kątem:
\begin{itemize}
    \item administracji
    \item obsługi dydaktyki
    \item komunikacji
    % \item integracji TODO 
    % TODO ew. dodać obsługę administracji - rodzice dokonują wpłat
\end{itemize}

W ramach zamówienia przygotowany zostanie system informatyczny wraz ze wsparciem technicznym i usługą serwisową. 
% pakiet serwisowy 


\section{Misja organizacji}
Założeniem istnienia Fundacji Nowoczesna Szkoła jest ... % TODO

dzieci i rodzice mają poczuć się ważną częścią placówki oświatowej\\
jak facebook - ma się stać częścią naszego życia\\
możliwość komunikowania się wiadomościami\\
w myśl zasady \textit{ oświata - like it!}\\
organizowanie wycieczek, szkolenia rozwijające \\
komunikacja komputerowa dla starszych pracowników szkół

\section{Opis zakresu projektu}
% wdrożenie w godz nocnych by nie zaburzyć pracy 
% precyzyjnie określić czas działania
% obsłużenie żądań 100 użytkowników w czasie


Projekt składa się z następujących modułów
\begin{itemize}
    \item moduł kadr
    \item moduł ocen
    \item moduł rodzica
    \item moduł dydaktyczny
\end{itemize}

Przedmiotem Umowy jest dostarczenie, wykonanie, instalacja i wdrożenie systemu \textit{eSzkoła}, obejmuje w szczególności:
%TODO dodać licencję
\begin{enumerate}
    \item wytworzenie, dostarczenie, zainstalowanie, skonfigurowanie i uruchomienie Systemu na zasobach sprzętowych wskazanych przez Zamawiającego, zgodnego z obowiązującym
    prawem;
    \item dostarczenie bazy danych zawierającej informacje dostarczone przez Zamawiającego w plikach zgodnych z formatem *.xls;
    \item skonfigurowanie i zintegrowanie dostarczanego Systemu do współpracy z systemami dzienniczka elektronicznego Librus i Vulcan
    a także zewnętrznego systemu kadr %TODO Flis team
\end{enumerate}

\subsection{Wymagania funkcjonalne}
\begin{enumerate}
    \item System pozwala zalogować się po podaniu poprawnego loginu i hasła
    \item Administrator może założyć nowe konto
    \item W systemie możliwe jest założenie kont użytkowników różnego typu
    \begin{itemize}
        \item \textbf{konto ucznia} - pozwala na dostęp do swoim danych osobowych, ocen cząstkowych i semestralnych, średniej ważonej ze wszystkich przedmiotów, planu zajęć, kalendarium szkoły, materiałów dydaktycznych udostępnionych przez prowadzących zajęcia
        \item \textbf{konto rodzica} - posiada uprawnienia konta ucznia rozszerzone o moduł opłat, możliwość umieszczania usprawiedliwień dla przyporządkowanych mu kont uczniowskich, oraz skrzynkę wiadomości do korespondencji z nauczycielami. Z każdym kontem rodzica związana musi być lista kont uczniowskich których prawnym opiekunem jest posiadacz tego konta.
        %TODO: Czy pisać o liście przedmiotów prowadzonych przez nauczyciela?
        \item \textbf{konto nauczyciela} - funkcjonalność konta rodzica ze zmodyfikowaną listą kont uczniowskich która zawiera wszystkich uczniów uczęszczających na prowadzone przedmioty, podzielona według przedmiotów i klas szkolnych. Dodatkowo konto to umożliwia zamieszczanie materiałów dydaktycznych, wystawianie ocen, kontakt z kontami rodzica, oraz uzupełnianie harmonogramu zajęć dla powiązanych kont ucznia.
        \item \textbf{konto administratora} - pozwala na zakładanie nowych kont użytkowników, tworzenie powiązań między nimi (np. przypisanie konta ucznia do odpowiadającego mu rodzica), modyfikację planu zajęć
        % TODO: Czy konto SUPERUSER jest nam niezbędne?
        \item \textbf{konto administratora głównego "SUPERUSER"} - konto o nieograniczonych przywilejach     
    \end{itemize}
    \item System oferuje integrację z istniejącymi rozwiązaniami dzienniczków elektronicznych LIBRUS oraz VULCAN na poziomie wyświetlania, modyfikacji oraz synchronizacji dwustronnej wprowadzonych ocen (tzn. wszelkie zmiany wprowadzone w dowolnym systemie powinny być odzwierciedlone w każdym z nich)
    \item Każde konto uprawnione do wystawiania ocen może to zrobić tylko w ramach swoich przedmiotów i tylko dla uczniów uczęszczających na te zajęcia
    \item Każda ocena powinna być wartością całkowitoliczbową z zakresu 1-6 z ewentualnym modyfikatorem w postaci symbolu "-" lub "+" występującym raz na samym końcu wartości oceny, posiadać przypisaną wagę (domyślnie ustawioną jako 1), zawierać krótką informację o powodzie jej wystawienia nie dłuższą niż 60 znaków a także datę wystawienia.
    \item Oceny wprowadzone do systemu mogą być usuwane oraz modyfikowane. Modyfikacja może obejmować zmiany w dowolnej wartości poza datą wystawienia oceny
    \item Możliwe jest wyliczenie średniej ważonej na podstawie zgromadzonych przez ucznia ocen i wyświetlenie jej jako proponowanej oceny końcowej
    \item Konto nauczyciela posiada możliwość zaakceptowania proponowanej oceny końcowej wyliczonej przez system jak również wprowadzenie własnej wartości
    \item Konta rodzica oraz ucznia mogą obejrzeć wystawione oceny a także wszystkie związane z nimi informacje
\end{enumerate}
%TODO podział funkcjonalnych:
system kadr: Dragan, Flis
system kadr: Pan Szef
system ocen: MW, KT
dydaktyka: K i H spis sprawdzianów, materiały dydaktyczne, plan zajęć, kalendarz, opis kont: uczeń ma możliwość podejrzenia, admin ma coś innego, nauczyciel może zgłosić prośbę o urlop, wypłaty, eksport cv (nauczyciel), elektroniczne wrzucanie zwolnień, zmiana wypłat, 


% dostarczenie bazy danych zawierającej informacje o planie zajęć, kołach naukowych oraz uroczystościach szkolnych dostarczonych przez Zamawiającego w formie elektronicznej zgodnej z formatem pliku *.xls;
\subsection{Wymagania niefunkcjonalne}
system ma działać w 99,9\% przez okres roku
% Marcin Kacper

\subsection{Sposób wdrożenia}
% Łukasz

\subsection{Warunki odbioru systemu}
\section{Utrzymanie systemu po wdrożeniu}
odporność na błędy, że działa przez jakiś okres czasu
%TODO warunki co do ludzi utrzymujących: wykształcenie, certyfikaty

\section{Dokumentacja systemu}
% PW
\section{Zasoby fundacji}
% TODO K & H

\section{Termin zakończenia}
% TODO K & H

\section{Kary umowne}
\end{document}
