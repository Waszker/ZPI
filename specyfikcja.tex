\documentclass{article}
\usepackage[MeX]{polski}
\usepackage[utf8]{inputenc}
\author{Fundacja nowoczesna szkoła\\\\
Honorata Rosłanowska \\
Marcin Wardziński\\
Kacper Sarnacki \\
Łukasz Dragan \\
Kacper Trojanowski \\
Mateusz Flis \\
Michał Grabowski \\
Piotr Waszkiewicz}
\title{eSzkoła - system zarządzania placówką oświatową}

\begin{document}
\maketitle
\newpage
\tableofcontents
\newpage

\section{Opis projektu}
Projekt eSzkoła zakłada stworzenie wielomodułowego systemu do zarządzania placówką oświatową. W ramach rozwiązania przewidziana jest realizacja modułów dla dyrektora, sekretariatu, nauczycieli, uczniów i ich rodziców pozwalających ułatwić dostęp i zarządzanie istotnymi danymi. \\ \\
Głównym celem zamawianego produktu jest usprawienie działania placówki oświatowej pod kątem:
\begin{itemize}
    \item administracji
    \item obsługi dydaktyki
    \item komunikacji
\end{itemize}
W ramach zamówienia przygotowany zostanie system informatyczny wraz ze wsparciem technicznym i usługą serwisową.

\section{Misja organizacji i cele projektu}
Fundacja Nowoczesna Szkoła została założona w 2006 roku jako organizacja wspierająca rozwój szkolnictwa. Jej celem jest ułatwienie dostępu do materiałów dydaktycznych, zapewnienie rodzicom lepszego nadzoru nad uczącym się dzieckiem a także skrócenie czasu potrzebnego na przygotowywanie dokumentów formalnych związanych z placówką oświatową.

dzieci i rodzice mają poczuć się ważną częścią placówki oświatowej\\
jak facebook - ma się stać częścią naszego życia\\
możliwość komunikowania się wiadomościami\\
w myśl zasady \textit{ oświata - like it!}\\
organizowanie wycieczek, szkolenia rozwijające \\
komunikacja komputerowa dla starszych pracowników szkół

\section{Opis zakresu projektu}
% wdrożenie w godz nocnych by nie zaburzyć pracy
% precyzyjnie określić czas działania
% obsłużenie żądań 100 użytkowników w czasie


Projekt składa się z następujących modułów
\begin{itemize}
    \item moduł kadr
    \item moduł ocen
    \item moduł rodzica
    \item moduł dydaktyczny
\end{itemize}

Przedmiotem Umowy jest dostarczenie, wykonanie, instalacja i wdrożenie systemu \textit{eSzkoła}, obejmuje w szczególności:
%TODO dodać licencję
\begin{enumerate}
    \item wytworzenie, dostarczenie, zainstalowanie, skonfigurowanie i uruchomienie Systemu na zasobach sprzętowych wskazanych przez Zamawiającego, zgodnego z obowiązującym
    prawem;
    \item dostarczenie bazy danych zawierającej informacje dostarczone przez Zamawiającego w plikach zgodnych z formatem *.xls;
    \item skonfigurowanie i zintegrowanie dostarczanego Systemu do współpracy z systemami dzienniczka elektronicznego Librus i Vulcan
    a także zewnętrznego systemu kadr %TODO Flis team
\end{enumerate}

\subsection{Wymagania funkcjonalne}
\begin{enumerate}
    \item System pozwala zalogować się po podaniu poprawnego loginu i hasła
    \item W systemie możliwe jest założenie kont użytkowników różnego typu:
    \begin{itemize}
        \item \textbf{konto ucznia}
        \begin{itemize}
            \item Zawiera informacje o imieniu, nazwisku, adresie ucznia oraz klasie, do której należy
            \item Pozwala na dostęp do planów zajęć wszystkich klas w szkole
            \item Umożliwia sprawdzenie swoich ocen cząstkowych i semestralnych, średniej ważonej ze wszystkich przedmiotów
            \item Udostępnia podgląd kalendarium szkoły
            \item Pozwala zapisać się do koła naukowego
            \item Umożliwia dostęp do prywatnego kalendarza zawierającego informacje o sprawdzianach, kołach naukowych, do których uczeń jest zapisany i zastępstwach nauczycieli, z którymi uczeń ma zajęcia
            \item Daje dostęp do materiałów dydaktycznych zamieszczanych przez nauczycieli
        \end{itemize}
        \item \textbf{konto rodzica} 
        \begin{itemize}
            \item Jest przypisane do co najmniej jednego ucznia
            \item Do jednego konta ucznia jest przypisane co najmniej jedno konto rodzica
            \item Ma dostęp do wszystkich danych, do których ma dostęp jego dziecko
            \item Umożliwia usprawiedliwienie nieobecności poprzez dodanie informacji tekstowej lub udostępnienie zeskanowanych zwolnień lekarskich
            \item Pozwala komunikować się z nauczycielami i innymi rodzicami poprzez wiadomości tekstowe
            \item Zawiera moduł opłat %TODO ?
        \end{itemize}
        %TODO: Czy pisać o liście przedmiotów prowadzonych przez nauczyciela?
        \item \textbf{konto nauczyciela} 
        \begin{itemize}
            \item Zawiera informacje o imieniu, nazwisku, adresie nauczyciela
            \item Jest przypisane do co najmniej jednego przedmiotu
            \item Zawiera listę wszystkich uczniów uczęszczających na prowadzone przedmioty, podzieloną według przedmiotów i klas szkolnych
            \item Pozwala komunikować się z rodzicami, innymi nauczycielami i administratorami poprzez wiadomości tekstowe
            \item Umożliwia wystawianie ocen cząstkowych i semestralnych z przedmiotów, do których jest przypisany nauczyciel
            \item Umożliwia dodawanie informacji o sprawdzianach
            \item Pozwala na dostęp do swojego planu zajęć oraz do wszystkich planów zajęć klas i nauczycieli
            \item Umożliwia udostępnianie materiałów dydaktycznych dowolnego typu wybranym klasom
        \end{itemize}
        \item \textbf{konto administratora}
        \begin{itemize}
             \item Pozwala na zakładanie nowych kont użytkowników i tworzenie powiązań między nimi (np. przypisanie konta ucznia do odpowiadającego mu rodzica)
             \item Daje możliwość dodawania, edycji i usuwania klas oraz zapisanych do nich uczniów 
             \item Umożliwia modyfikowanie planu zajęć
             \item Pozwala na dodawanie i usuwanie z kalendarza informacji o nieobecności nauczycieli oraz wycieczkach szkolnych
             \item Umożliwia zmianę ocen uczniów
         \end{itemize} 
        % TODO: Czy konto SUPERUSER jest nam niezbędne?
        %\item \textbf{konto administratora głównego \"SUPERUSER\"} - konto o nieograniczonych przywilejach
    \end{itemize}
    % TODO: Issue #001
    \item Materiały dydaktyczne są sposobem na komunikację między nauczycielem a uczniami. Są to pliki dowolnego typu udostępniane w aplikacji. Nauczyciel ma możliwość dodawania i usuwania ich, uczeń ma możliwośc dodawania. Nauczyciel ma możliwość udostępniania materiałów wszystkim uczniom należącym do klasy, którą uczy. Jednorazowo wrzucone dane nie powinny mieć wielkości przekraczającej 10 MB.
    % TODO: Issue #002
    \item Plan zajęć jest przypisany do klasy albo nauczyciela. Zawiera informacje o przedmiotach: godzinach ich rozpoczęcia i zakończenia, sali, w której się odbywają, nauczycielu prowadzącym zajęcia (w przypadku planu ucznia).
    \item Kalendarium zawiera informacje o wydarzeniach szkolnych takich jak apele i uroczystości, a także dniach wolnych od zajęć dydaktycznych. Każde takie wydarzenie oprócz swojej daty posiada informację o godzinie rozpoczęcia i zakończenia, lokalizacji oraz krótką informację w postaci notki nie dłuższej niż 100 znaków.
    \item Kalendarz ucznia jest zbiorem spersonalizowany wydarzeń każdego ucznia. Zawiera informacje o zastępstwach nauczycieli, z którymi uczeń ma zajęcia, informacje o sprawdzianach (ich data oraz przedmiot) oraz o kołach naukowych, do których zapisany jest uczeń
    % TODO: Issue #003
    \item Klasa będąca zbiorem uczniów, posiada swoją nazwę oznaczaną jedną dużą literą alfabetu łacińskiego. Ma przypisanego wychowawcę (dokładnie jednego nauczyciela), oraz plan zajęć
    \item System oferuje integrację z istniejącymi rozwiązaniami dzienniczków elektronicznych LIBRUS oraz VULCAN na poziomie wyświetlania, modyfikacji oraz synchronizacji dwustronnej wprowadzonych ocen (tzn. wszelkie zmiany wprowadzone w dowolnym systemie powinny być odzwierciedlone w każdym z nich)
    % TODO: Co w przypadku konta administratora? Czy poniższe wymaganie jest wystarczająco dokładnie napisane?
    \item Każde konto uprawnione do wystawiania ocen może to zrobić tylko w ramach swoich przedmiotów i tylko dla uczniów uczęszczających na te zajęcia
    \item Każda ocena powinna być wartością całkowitoliczbową z zakresu 1-6 z ewentualnym modyfikatorem w postaci symbolu \"-\" lub \"+\" występującym raz na samym końcu wartości oceny, posiadać przypisaną wagę (domyślnie ustawioną jako 1), zawierać krótką informację o powodzie jej wystawienia nie dłuższą niż 60 znaków a także datę wystawienia.
    \item Oceny wprowadzone do systemu mogą być usuwane oraz modyfikowane. Modyfikacja może obejmować zmiany dowolnej wartości poza datą wystawienia oceny
    \item Możliwe jest wyliczenie średniej ważonej na podstawie zgromadzonych przez ucznia ocen i wyświetlenie jej jako proponowanej oceny końcowej
    \item Konto nauczyciela posiada możliwość zaakceptowania proponowanej oceny końcowej wyliczonej przez system jak również wprowadzenie własnej wartości
    \item Konta rodzica oraz ucznia mogą obejrzeć wystawione oceny a także wszystkie związane z nimi informacje
\end{enumerate}
%TODO podział funkcjonalnych:
system kadr: Dragan, Flis

\paragraph{System kadr}
\begin{itemize}


\item Zgłoszenie wniosku urlopowego\\
Inicjator: nauczyciel
\begin{enumerate}
    \item Nauczyciel zgłasza potrzebę złożenia wniosku urlopowego
    \item System udostępnia możliwość podania zakresu urlopu
    \item Nauczyciel wprowadza dane do systemu i zatwierdza zmiany
    \item System zapisuje zmiany
\end{enumerate}
    \item Zatwierdzenie wniosku urlopowego
Inicjator: system
\begin{enumerate}
    \item Administrator otrzymuje powiadomienie o wniosku urlopowym
    \item Administrator zatwierdza/odrzuca wniosek
    \item System informuje nauczyciela o statusie wniosku
\end{enumerate}
\item Ustalenie zastępstwa
Inicjator: administrator
\begin{enumerate}
    \item Administrator zgłasza potrzebę zastąpieniu nauczyciela na lekcjach
    \item System udostępnia możliwość podania zakresu zastępstwa
    \item Administrator wprowadza dane
    \item System udostępnia informację o możliwej konfiguracji zastępstw
    \item Administrator wybiera satysfakcjonujący model zastępstwa
    \item System zapisuje zmiany i powiadamia nauczycieli o tym, iż są proszeni o uczestnictwo w zastępstwie
\end{enumerate}
    \item Integracja z systemem kadrowo-płacowym
Inicjator: system
\begin{enumerate}
\item 
\end{enumerate}

\item Wyszukanie pracownika w systemie
Inicjator: administrator
\begin{enumerate}
    \item Administrator zgłasza potrzebę wyszukania pracownika w systemie
    \item System udostępnia możliwość wyszukiwania wśród pracowników względem danych użytkownika
    \item Administrator wprowadza szukaną frazę
    \item System udostępnia listę najtrafniejszych wyników wyszukiwania
    \item Nauczyciel wybiera konkretny wynik
    \item System udostępnia szczegółowe dane pracownika
\end{enumerate}

\item Zatrudnienie nowego pracownika
Inicjator: administrator
\begin{enumerate}
    \item Administrator zgłasza potrzebę dodania nowego pracownika do systemu
    \item System udostępnia możliwość podania jego danych osobowych
    \item Administrator wprowadza dane do systemu i zatwierdza wybór
    \item System zapisuje dane
\end{enumerate}
\item Zwolnienie pracownika
Inicjator: administrator
\begin{enumerate}
    \item Administrator zgłasza potrzebę zmiany statusu pracownika na ,,zwolniony''
    \item System udostępnia możliwość wyszukania pracownika
    \item Administrator zmienia status pracownika zatwierdza wybór
    \item System zapisuje dane
\end{enumerate}
\end{itemize}









system kadr: Pan Szef
system ocen: MW, KT
dydaktyka: K i H spis sprawdzianów, materiały dydaktyczne, plan zajęć, kalendarz, opis kont: uczeń ma możliwość podejrzenia, admin ma coś innego, nauczyciel może zgłosić prośbę o urlop, wypłaty, eksport cv (nauczyciel), elektroniczne wrzucanie zwolnień, zmiana wypłat,


% dostarczenie bazy danych zawierającej informacje o planie zajęć, kołach naukowych oraz uroczystościach szkolnych dostarczonych przez Zamawiającego w formie elektronicznej zgodnej z formatem pliku *.xls;
\subsection{Wymagania niefunkcjonalne}
system ma działać w 99,9\% przez okres roku
% Marcin Kacper

\subsection{Sposób wdrożenia}
% Łukasz
\paragraph{Etapy}
\begin{enumerate}
\item 
\end{enumerate}
\paragraph{Przebieg szkoleń}
W ramach wdrożenia systemu odbędzie się cykl szkoleń przeznaczonych dla uczniów, rodziców, nauczycieli i administratorów. Szkolenia będą prowadzone w salach szkolnych przeznaczonych do nauki informatyki - wyposażone w stanowiska komputerowe, projektor oraz tablicę do rysowania.
\begin{itemize}
\item Szkolenia dla uczniów odbywają się w ramach przedmiotu ,,informatyka''.
\item Szkolenia dla chętnych rodziców odbywają się raz na grupę osób w godzinach 17:00 - 19:00.
\item Szkolenia dla pracowników szkoły odbywają się w godzinach 09:00-12:00.
\end{itemize}
Pracownicy oddelegowani do udziału w szkoleniach obowiązani są uczestniczyć w nich w pełnym wymiarze czasu. Po zakończeniu szkoleń wszyscy uczestnicy wypełniają ,,Ankietę oceny szkolenia''.

Do systemu dołączony zostanie ,,samouczek'' - dokument opisujący działanie poszczególnych części systemu.

\subsection{Warunki odbioru systemu}
\section{Utrzymanie systemu po wdrożeniu}
odporność na błędy, że działa przez jakiś okres czasu
%TODO warunki co do ludzi utrzymujących: wykształcenie, certyfikaty

\section{Dokumentacja systemu}
% PW
\section{Zasoby}
Oświadczamy, iż dysponujemy następującymi zasobami:
\begin{itemize}
    \item Pliki w formacie *.xls zawierające wszystkie plany zajęć klas oraz nauczycieli
    \item Pliki w formacie *.xls zawierające spis uczniów z informacjami: imię, nazwisko, data urodzenia, pesel, adres, data rozpoczęcia szkoły, klasa, do której uczeń jest zapisany
    \item Pliki w formacie *.xls zawierające spis opiekunów uczniów z informacjami: imię i nazwisko opiekuna, pesel ucznia, adres, telefon kontaktowy 
    \item Pliki w formacie *.xls zawierające spis nauczycieli z informacjami: imię, nazwisko, adres, telefon kontaktowy, konto bankowe %TODO czy uwzględniamy wypłaty nauczycieli?
\end{itemize}

Wykonujący musi dysponować następującymi zasobami osobowymi:
\begin{itemize}
    \item co najmniej dwóch specjalistów baz danych o nie mniej niż dwuletnim doświadczeniu na podobnym stanowisku
    \item co najmniej czterech programistów posiadających certyfikat potwierdzający ich umiejętności programowania w języku Fortran
\end{itemize}


\section{Harmonogram wykonania Umowy}
\begin{enumerate}
    \item Umowa będzie realizowana od daty jej zawarcia i w okresie nie dłuższym niż do dnia 30 czerwca 2018 r., zgodnie ze szczegółowym Harmonogramem
    \item Wykonawca w terminie do 10 dni od dnia zawarcia Umowy, wytworzy i dostarczy do akceptacji Zamawiającego Harmonogram uwzględniający terminy i etapy zawarte w Umowie. Wykonawca uwzględni uwagi Zamawiającego dotyczące opracowanego Harmonogramu prac
    \item Wykonawca jest zobowiązany do realizacji przedmiotu Umowy w sposób następujący:
    \begin{enumerate}
        \item etap 1 - analiza przedwdrożeniowa - opracowanie projektu technicznego systemu eSzkoła w terminie do 30 dni od dnia zawarcia Umowy
        \item etap 2 - implementacja systemu w terminie do 275 dni od dnia zawarcia      Umowy nie później jednak niż do dnia 20 czerwca 2017 r.
        \item etap 3 - integracja systemu eSzkoła z istniejącymi rozwiązaniami dzienniczków elektronicznych LIBRUS oraz VULCAN w terminie nie późniejszym niż do dnia 20 czerwca 2017 r.
        \item Po realizacji etapów 1, 2 i 3 Wykonawca zobowiązuje się do wsparcia technicznego oraz zapewnienia usług serwisowych przez okres roku nie później niż do 30 czerwca 2018 r.
    \end{enumerate}
\end{enumerate}

\section{Kary umowne}
\end{document}
