\documentclass{article}
\usepackage[MeX]{polski}
\usepackage[utf8]{inputenc}
\author{Fundacja nowoczesna szkoła\\\\
Honorata Rosłanowska \\
Marcin Wardziński\\
Kacper Sarnacki \\
Łukasz Dragan \\
Kacper Trojanowski \\
Mateusz Flis \\
Michał Grabowski \\
Piotr Waszkiewicz}
\title{eSzkoła - system zarządzania placówką oświatową}

\begin{document}
\maketitle
\newpage
\tableofcontents
\newpage

\section{Opis projektu}
Projekt eSzkoła zakłada stworzenie wielomodułowego systemu do zarządzania placówką oświatową. W ramach rozwiązania przewidziana jest realizacja modułów dla dyrektora, sekretariatu, nauczycieli, uczniów i ich rodziców pozwalających ułatwić dostęp i zarządzanie istotnymi danymi. \\ \\
Głównym celem zamawianego produktu jest usprawienie działania placówki oświatowej pod kątem:
\begin{itemize}
    \item administracji
    \item obsługi dydaktyki
    \item komunikacji
\end{itemize}
W ramach zamówienia przygotowany zostanie system informatyczny wraz ze wsparciem technicznym i usługą serwisową.

\section{Misja organizacji i cele projektu}
Fundacja Nowoczesna Szkoła została założona w 2006 roku jako organizacja wspierająca rozwój szkolnictwa. Jej celem jest ułatwienie dostępu do materiałów dydaktycznych, zapewnienie rodzicom lepszego nadzoru nad uczącym się dzieckiem a także skrócenie czasu potrzebnego na przygotowywanie dokumentów formalnych związanych z placówką oświatową. Szkoła jest miejscem w którym młodzi ludzie spędzają dużą część swojego czasu. Dotyczy to zarówno zajęć jak i czasu poświęconego na aktywności związane z przedmiotami szkolnymi poza jej murami. Uczniowie oprócz odrabiania lekcji muszą pamiętać o sprawdzianach, opłatach i wydarzeniach które organizuje szkoła. Comiesięczne zebrania z rodzicami nie pozwalają nauczycielom w wystarczającym stopniu przekazać informacji o postępach w kształceniu ich pociech. \\ \\
Wierzymy, że wraz z rozwojem platformy eSzkoła sytuacja ta ulegnie znacznej poprawie. Możliwość sprawdzenia ocen, zmian w planie lekcji oraz dostęp do materiałów udostępnionych przez prowadzących zajęcia przy użyciu komputera pozwoli na sprawniejsze kontrolowanie istotnych informacji. Elektroniczny system dzienniczka umożliwi nauczycielom natychmiastowy kontakt z rodzicami gdy zajdzie taka potrzeba. Moduł opłat przyspieszy dokonywanie przez rodziców przelewów a system kadr dostępny dla dyrektora i sekretariatu ułatwi kontrolowania wydatków placówki oświatowej. Możliwość rejestrowania wniosków o urlop, zwolnień lekarskich oraz podań usprawni przepływ dokumentów formalnych.

\newpage
\section{Opis zakresu projektu}
% wdrożenie w godz nocnych by nie zaburzyć pracy
% precyzyjnie określić czas działania
% obsłużenie żądań 100 użytkowników w czasie


Projekt składa się z następujących modułów
\begin{itemize}
    \item moduł dydaktyki
    \item moduł ocen
    \item moduł kont
    \item moduł administracji
\end{itemize}

Przedmiotem Umowy jest dostarczenie, wykonanie, instalacja i wdrożenie systemu \textit{eSzkoła}, obejmuje w szczególności:
%TODO dodać licencję
\begin{enumerate}
    \item wytworzenie, dostarczenie, zainstalowanie, skonfigurowanie i uruchomienie Systemu na zasobach sprzętowych wskazanych przez Zamawiającego, zgodnego z obowiązującym prawem;
    \item dostarczenie bazy danych zawierającej informacje dostarczone przez Zamawiającego w plikach zgodnych z formatem *.xls;
    \item skonfigurowanie i zintegrowanie dostarczanego Systemu do współpracy z systemami dzienniczka elektronicznego Librus i Vulcan a także zewnętrznego systemu kadr
\end{enumerate}

\subsection{Wymagania funkcjonalne}

\paragraph{Moduł dydaktyki} \mbox{}\\

\paragraph{Moduł ocen} \mbox{}\\
\begin{enumerate}
	\item Każde konto uprawnione do wystawiania ocen może to zrobić tylko w ramach swoich przedmiotów i tylko dla uczniów uczęszczających na te zajęcia.
    \item Każda ocena powinna być wartością całkowitoliczbową z zakresu 1-6 z ewentualnym modyfikatorem w postaci symbolu "-" lub "+" występującym raz na samym końcu wartości oceny, posiadać przypisaną wagę (domyślnie ustawioną jako 1), zawierać krótką informację o powodzie jej wystawienia nie dłuższą niż 60 znaków a także datę wystawienia.
    \item Oceny wprowadzone do systemu mogą być usuwane oraz modyfikowane. Modyfikacja może obejmować zmiany dowolnej wartości poza datą wystawienia oceny.
    \item Każde konto uprawnione do wystawiania ocen może wstawiać także punkty za aktywność lub nieprzygotowanie w postaci symboli "+" i "-". Takie punkty powinny zachować zasady wyświetlania i modyfikacji zgodnie z zasadami ocen.
	\item Każde trzy punkty "+" i każde trzy punkty "-" powinny mieć możliwość automatycznej zamiany na odpowiednio oceny 5 i 1 z przypisaną wagą oceny 1.
    \item Możliwe jest wyliczenie średniej ważonej na podstawie zgromadzonych przez ucznia ocen i wyświetlenie jej jako proponowanej oceny końcowej.
    \item Konto nauczyciela posiada możliwość zaakceptowania proponowanej oceny końcowej wyliczonej przez system jak również wprowadzenie własnej wartości.
    \item Konta rodzica oraz ucznia mogą obejrzeć wystawione oceny a także wszystkie związane z nimi informacje.
\end{enumerate}

\paragraph{Moduł kont} \mbox{}\\

\paragraph{Moduł administracji} \mbox{}\\

\begin{enumerate}
    \item System pozwala zalogować się po podaniu poprawnego loginu i hasła
    \item W systemie możliwe jest założenie kont użytkowników różnego typu:
    \begin{itemize}
        \item \textbf{konto ucznia}
        \begin{itemize}
            \item Zawiera informacje o imieniu, nazwisku, adresie ucznia oraz klasie, do której należy
            \item Umożliwia sprawdzenie swoich ocen cząstkowych i semestralnych, średniej ważonej ze wszystkich przedmiotów
            \item Udostępnia podgląd kalendarium szkoły
            \item Pozwala zapisać się do koła naukowego
            \item Umożliwia dostęp do prywatnego kalendarza zawierającego informacje o sprawdzianach, kołach naukowych, do których uczeń jest zapisany i zastępstwach nauczycieli, z którymi uczeń ma zajęcia
            \item Daje dostęp do materiałów dydaktycznych zamieszczanych przez nauczycieli
        \end{itemize}
        \item \textbf{konto rodzica} 
        \begin{itemize}
            \item Jest przypisane do co najmniej jednego ucznia
            \item Do jednego konta ucznia jest przypisane co najmniej jedno konto rodzica
            \item Ma dostęp do wszystkich danych, do których ma dostęp jego dziecko
            \item Umożliwia usprawiedliwienie nieobecności poprzez dodanie informacji tekstowej lub udostępnienie zeskanowanych zwolnień lekarskich
            \item Pozwala komunikować się z nauczycielami i innymi rodzicami poprzez wiadomości tekstowe
            \item Zawiera moduł opłat %TODO ?
        \end{itemize}
        %TODO: Czy pisać o liście przedmiotów prowadzonych przez nauczyciela?
        \item \textbf{konto nauczyciela} 
        \begin{itemize}
            \item Zawiera informacje o imieniu, nazwisku, adresie nauczyciela
            \item Jest przypisane do co najmniej jednego przedmiotu
            \item Zawiera listę wszystkich uczniów uczęszczających na prowadzone przedmioty, podzieloną według przedmiotów i klas szkolnych
            \item Pozwala komunikować się z rodzicami, innymi nauczycielami i administratorami poprzez wiadomości tekstowe
            \item Umożliwia wystawianie ocen cząstkowych i semestralnych z przedmiotów, do których jest przypisany nauczyciel
            \item Umożliwia dodawanie informacji o sprawdzianach
            \item Umożliwia udostępnianie materiałów dydaktycznych dowolnego typu wybranym klasom
        \end{itemize}
        \item \textbf{konto administratora}
        \begin{itemize}
             \item Pozwala na zakładanie nowych kont użytkowników i tworzenie powiązań między nimi (np. przypisanie konta ucznia do odpowiadającego mu rodzica)
             \item Daje możliwość dodawania, edycji i usuwania klas oraz zapisanych do nich uczniów 
             \item Umożliwia modyfikowanie planu zajęć
             \item Pozwala na dodawanie i usuwanie z kalendarza informacji o nieobecności nauczycieli oraz wycieczkach szkolnych
             \item Umożliwia zmianę ocen uczniów
         \end{itemize} 
        % TODO: Czy konto SUPERUSER jest nam niezbędne?
        %\item \textbf{konto administratora głównego \"SUPERUSER\"} - konto o nieograniczonych przywilejach
    \end{itemize}
    \item Materiały dydaktyczne to pliki dowolnego typu udostępniane w aplikacji. Nauczyciel ma możliwość pobierania, dodawania i usuwania ich, uczeń ma możliwość pobierania ich. Nauczyciel ma możliwość udostępniania materiałów na stronę przedmiotu, którego uczy. Jednorazowo wrzucone dane nie mogą przekraczać 10 MB.
    % TODO: Issue #002
    \item Plan zajęć klasy zawiera informacje o przedmiotach: godzinach ich rozpoczęcia i zakończenia, sali, w której się odbywają oraz nauczycielu prowadzącym zajęcia. Może być wyświetlany przez dowolny typ konta.
    \item Plan zajęć nauczyciela zawiera informacje o przedmiotach: godzinach ich rozpoczęcia i zakończenia, sali, w której się odbywają oraz klasie, z którą ma zajęcia. Może być wyświetlany przez dowolny typ konta.
    \item Kalendarium zawiera informacje o wydarzeniach szkolnych takich jak apele i uroczystości, a także dniach wolnych od zajęć dydaktycznych. Każde takie wydarzenie oprócz swojej daty posiada informację o godzinie rozpoczęcia i zakończenia, lokalizacji oraz krótką informację w postaci notki nie dłuższej niż 100 znaków.
    \item Kalendarz ucznia jest zbiorem spersonalizowany wydarzeń każdego ucznia. Zawiera informacje o zastępstwach nauczycieli, z którymi uczeń ma zajęcia, informacje o sprawdzianach (ich data oraz przedmiot) oraz o kołach naukowych, do których zapisany jest uczeń
    \item Klasa będąca zbiorem uczniów, posiada swoją nazwę oznaczaną jedną cyfrą (rok nauki) oraz jedną dużą literą alfabetu łacińskiego. Ma przypisanego wychowawcę (dokładnie jednego nauczyciela), oraz plan zajęć
    \item System oferuje integrację z istniejącymi rozwiązaniami dzienniczków elektronicznych LIBRUS oraz VULCAN na poziomie wyświetlania, modyfikacji oraz synchronizacji dwustronnej wprowadzonych ocen (tzn. wszelkie zmiany wprowadzone w dowolnym systemie powinny być odzwierciedlone w każdym z nich)
    % TODO: Co w przypadku konta administratora? Czy poniższe wymaganie jest wystarczająco dokładnie napisane?
    \item Każde konto uprawnione do wystawiania ocen może to zrobić tylko w ramach swoich przedmiotów i tylko dla uczniów uczęszczających na te zajęcia
    \item Każda ocena powinna być wartością całkowitoliczbową z zakresu 1-6 z ewentualnym modyfikatorem w postaci symbolu "-" lub "+" występującym raz na samym końcu wartości oceny, posiadać przypisaną wagę (domyślnie ustawioną jako 1), zawierać krótką informację o powodzie jej wystawienia nie dłuższą niż 60 znaków a także datę wystawienia.
    \item Oceny wprowadzone do systemu mogą być usuwane oraz modyfikowane. Modyfikacja może obejmować zmiany dowolnej wartości poza datą wystawienia oceny
    \item Możliwe jest wyliczenie średniej ważonej na podstawie zgromadzonych przez ucznia ocen i wyświetlenie jej jako proponowanej oceny końcowej
    \item Konto nauczyciela posiada możliwość zaakceptowania proponowanej oceny końcowej wyliczonej przez system jak również wprowadzenie własnej wartości
    \item Konta rodzica oraz ucznia mogą obejrzeć wystawione oceny a także wszystkie związane z nimi informacje
\end{enumerate}

\paragraph{Moduł rodzica} \mbox{}\\
Po zalogowaniu się jako rodzic:
\begin{enumerate}
	\item Użytkownik ma możliwość zmiany ustawień swojego konta. Użytkownik ma możliwość dokonania konfiguracji następujących elementów:
	\begin{enumerate}
		\item Numer telefonu
		\item Adres email
		\item Adres do korespondencji
		\item Numer konta bankowego
		\item Powiadomienia wysyłane przez aplikację tj.:
		\begin{enumerate}
			\item Użytkownik powinien mieć wybór sposobu, w którym otrzymywać będzie powiadomienia (SMS/email)
			\item Użytkownik powinien mieć wybór powiadomień, które będzie otrzymywać. Możliwość konfiguracji powinna dotyczyć osobno każdego ze sposobów przesyłania powiadomień oraz każdego z kont Ucznia, które są powiązane z kontem Rodzica. Powiadomienia, które mogą być przesyłane do Użytkownika są następujące:
			\begin{enumerate}
				\item powiadomienia o ocenach (możliwość filtrowania ocen, które powodują wysłanie powiadomienia)
				\item powiadomienia o nieobecnościach (możliwość wyboru nieobecności na sprawdzianach, zajęciach pozalekcyjnych)
				\item powiadomienia o spóźnieniach
				\item powiadomienia o pojawieniu się nowych opłat
				\item powiadomienia o pojawieniu się nowej ankiety
				\item powiadomienia o nadchodzących terminach opłat (możliwość w wyboru terminu przed terminem opłaty, kiedy pojawiać będzie się powiadomienie)
				\item powiadomienia o zebraniach rodziców
				\item powiadomienia o nowych wydarzeniach i uroczystościach szkolnych
				\item powiadomienia o przyznanych stypendiach
				\item powiadomienia o otrzymanych świadectwach
				\item powiadomienia o nadchodzących egzaminach
			\end{enumerate}
		\end{enumerate} 
	\end{enumerate}
	\item System umożliwia wybór konta Ucznia, dla którego wdrożone będą poniższe funkcjonalności modułu Rodzica w przypadku, gdy do jednego konta Rodzica przypisane będzie więcej niż jedno konto ucznia
	\item Rodzic będzie miał możliwość podglądu osiągnięć ucznia, uwag dotyczących jego sprawowania. Rodzic będzie miał wgląd do następujących elementów:
	\begin{enumerate}
		\item Oceny
		\item Obecności
		\item Spóźnienia
	\end{enumerate}
	\item Ponadto rodzic będzie miał dostęp do informacji o szczególnych osiągnięciach ucznia:
	\begin{enumerate}
		\item 	Wygrane w konkursach naukowych
		\item Wygrane w zawodach sportowych
		\item Osiągnięcia artystyczne
	\end{enumerate}
	\item Rodzic będzie miał możliwość podglądu planu zajęć Ucznia z naniesionymi na niego aktualny zmianami  (np. uroczystości szkolne, nieobecności nauczycieli)
	\item Rodzic będzie miał podawane informacje o terminach planowanych prac klasowych Ucznia oraz o zakresie materiału na nich obowiązującym
	\item Rodzic za pośrednictwem systemu będzie miał możliwość usprawiedliwiania oraz zwalniania podopiecznego z zajęć lekcyjnych
	\item Rodzic będzie zaznajomiony z informacjami o samorządzie klasowym ucznia - o pełnionych przez Uczniów funkcjach i zakresie ich obowiązków
	\item Rodzic będzie miał wgląd do wydarzeń organizowanych w szkole wliczając w to:
	\begin{enumerate}
		\item uroczystości szkolne
		\item organizowane wycieczki
		\item wydarzenia sportowe
	\end{enumerate}
	\item W panelu rodzica zamieszczone będą informacje na temat godzin dyżurów nauczycieli, w trakcie których Rodzic będzie mógł porozmawiać na temat swojego podopiecznego
	\item Moduł Rodzica umożliwi widok historycznych oraz obecnych opłat i należności przypisanych do danego konta
	\item Opłaty będą mogły być uregulowane za pośrednictwem Systemu. Funkcjonalność ta będzie wprowadzona poprzez integrację z systemem płatności elektronicznych za pomocą przelewów PayU, kart kredytowych/debetowych eService oraz systemem PayPal
	\item Rodzic będzie miał możliwość wygenerowania dowodów wniesienia opłaty w formacie PDF
	\item Rodzic będzie widział stypendia przyznane Uczniowi. Stypendia wpłacane będą na konto bankowe, które podał Rodzic podczas konfiguracji swojego konta użytkownika
	\item Świadectwa otrzymane przez ucznia widoczne będą w Panelu Rodzica
	\item Rodzic będzie miał dostęp do informacji o kołach naukowych (zainteresowań) do których należy Uczeń. Rodzic będzie miał również podgląd płatnych zajęć pozalekcyjnych oraz możliwość zapisania na nie Ucznia
	\item Zebrania rodziców zwoływane przez nauczycieli lub dyrekcję będą wyświetlane w panelu Rodzica.  Rodzic będzie miał możliwość potwierdzenia swojej obecności, lub poinformowaniu o swojej absencji
	\item Katalog przedmiotów obowiązkowych w przebiegu edukacji szkolnej dla danego ucznia będzie udostępniony Rodzicowi, zawarte w nim informacje dotyczące pojedynczego przedmiotu będą następujące:
	\begin{enumerate}
		\item Liczba godzin lekcyjnych
		\item Zakres materiału
		\item Realizowane wymagania egzaminacyjne 
	\end{enumerate}
	\item Poprzez panel Rodzic będzie miał dostęp do informatorów udostępnianych przez CKE dotyczących egazminu, które zdawać będzie uczeń kończąc szkołę 
	\item Rodzic będzie miał możliwość uzupełniania ankiet przygotowanych przez administrację szkolną oraz podglądu wyłącznie udostępnionych przez administrację wyników
	\item System umożliwi przesyłanie powiadomień informujących o zmianach pojawiających się w informacjach zawartych w panelu Rodzica. Powiadomienia będą dotyczyć:
	\begin{enumerate}
		\item ocen ucznia
		\item nieobecności oraz spóźnień na zajęcia
		\item nowych opłat
		\item terminów opłat 
		\item ankiet
		\item zebrań rodziców
		\item wydarzeń i uroczystości szkolnych
		\item przyznanych stypendiów
		\item otrzymanych świadectw
		\item nadchodzących egzaminów
	\end{enumerate}	
\end{enumerate}

Dodatkowo Rodzic może mieć rozszerzoną funkcjonalność konta w przypadku gdy jest przedstawicielem Komitetu Rodzicielskiego. W takim przypadku dodatkowo posiada możliwość:
\begin{enumerate}
	\item Organizowania zbiórki pieniężnej na cele komitetu rodzicielskiego
	\item Układania ankiet skierowanych do pozostałych rodziców w klasie ucznia
	\item Zwoływanie zebrań Komitetu Rodzicielskiego
\end{enumerate}

\paragraph{Moduł ucznia} \mbox{}\\
Po zalogowaniu się jako uczeń:
\begin{enumerate}
	\item Użytkownik ma możliwość zmiany ustawień swojego konta. Użytkownik ma możliwość dokonania konfiguracji następujących elementów:
	\begin{enumerate}
		\item Numer telefonu
		\item Adres email
		\item Adres do korespondencji
		\item Powiadomienia wysyłane przez aplikację tj.:
		\begin{enumerate}
			\item Użytkownik powinien mieć wybór sposobu, w którym otrzymywać będzie powiadomienia (SMS/email)
			\item Użytkownik powinien mieć wybór powiadomień, które będzie otrzymywać. Możliwość konfiguracji powinna dotyczyć osobno każdego ze sposobów przesyłania powiadomień. Powiadomienia, które mogą być przesyłane do Użytkownika są następujące:
			\begin{enumerate}
				\item powiadomienia o wystawionych ocenach
				\item powiadomienia o pojawieniu się nowej ankiety
				\item powiadomienia o zebraniach rodziców
				\item powiadomienia o nowych wydarzeniach i uroczystościach szkolnych
				\item powiadomienia o przyznanych stypendiach
				\item powiadomienia o otrzymanych świadectwach
				\item powiadomienia o nadchodzących egzaminach
			\end{enumerate}
		\end{enumerate} 
	\end{enumerate}
	\item Uczeń będzie miał możliwość podglądu swoich osiągnięć, oraz uwag dotyczących jego sprawowania. Będzie miał wgląd do następujących elementów:
	\begin{enumerate}
		\item Oceny
		\item Obecności
		\item Spóźnienia
	\end{enumerate}
	\item Ponadto uczeń będzie miał dostęp do informacji o swoich szczególnych osiągnięciach:
	\begin{enumerate}
		\item Wygrane w konkursach naukowych
		\item Wygrane w zawodach sportowych
		\item Osiągnięcia artystyczne
	\end{enumerate}
	\item Uczeń będzie miał możliwość podglądu planu zajęć z naniesionymi na niego aktualny zmianami  (np. uroczystości szkolne, nieobecności nauczycieli)
	\item Uczeń będzie miał podawane informacje o terminach planowanych prac klasowych oraz o zakresie materiału na nich obowiązującym
	\item Uczeń będzie zaznajomiony z informacjami o samorządzie klasowym - o pełnionych przez niego funkcjach i zakresie obowiązków
	\item Uczeń będzie miał wgląd do wydarzeń organizowanych w szkole wliczając w to:
	\begin{enumerate}
		\item uroczystości szkolne
		\item organizowane wycieczki
		\item wydarzenia sportowe
	\end{enumerate}
	\item W panelu ucznia zamieszczone będą informacje na temat godzin dyżurów nauczycieli
	\item Uczeń będzie widział przyznane mu stypendia oraz wystawione świadectwa
	\item Uczeń będzie miał dostęp do informacji o kołach naukowych (zainteresowań) do których należy.
	\item Katalog przedmiotów obowiązkowych w przebiegu edukacji szkolnej będzie udostępniony Uczniowi, zawarte w nim informacje dotyczące pojedynczego przedmiotu będą następujące:
	\begin{enumerate}
		\item Liczba godzin lekcyjnych
		\item Zakres materiału
		\item Realizowane wymagania egzaminacyjne 
	\end{enumerate}
	\item Poprzez panel Uczeń będzie miał dostęp do informatorów udostępnianych przez CKE dotyczących egazminu, które zdawać będzie kończąc szkołę 
	\item Uczeń będzie miał możliwość uzupełniania ankiet przygotowanych przez administrację szkolną oraz podglądu wyłącznie udostępnionych przez administrację wyników
	\item System umożliwi przesyłanie powiadomień informujących o zmianach pojawiających się w informacjach zawartych w panelu Uczeń. Powiadomienia będą dotyczyć:
	\begin{enumerate}
		\item ocen ucznia
		\item nieobecności oraz spóźnień na zajęcia
		\item nowych opłat
		\item terminów opłat 
		\item ankiet
		\item zebrań rodziców
		\item wydarzeń i uroczystości szkolnych
		\item przyznanych stypendiów
		\item otrzymanych świadectw
		\item nadchodzących egzaminów
	\end{enumerate}	
\end{enumerate}

\paragraph{System administracji} służy zarządzaniu użytkownikami systemu a także spełnia rolę systemu kard, który umożliwia prowadzenie ewidencji pracowników szkoły.\\Dostęp do modułu mają jedynie wybrane osoby z administracji placówki oświatowej posiadające status administratora systemu. Konto administratora ma dostęp do następujących czynności i informacji oraz składa się z elementów wyszczególnionych w poniższym opisie:
\begin{enumerate}
\item W systemie administracji znajduje się lista wszystkich użytkowników, których administrator może swobodnie wyszukiwać i filtrować.
\item System pozwala na dodanie nowego konta użytkownika do systemu: ucznia, nauczyciela lub rodzica. Przy tej operacji niezbędne jest podanie podstawowych danych ucznia, które będą dostępne w profilu ucznia.
\item Konto administratora umożliwia edycję danych personalnych dowolnego użytkownika systemu.
\item Istnieje możliwość usunięcia dowolnego użytkownika systemu.
\item System oferuje możliwość edycji ocen wystawionych przez nauczyciela.

\item Administrator może zmienić status pracy nauczyciela: aktywny, urlopowany, niepracujący, zwolniony, emerytowany, na zwolnieniu lekarskim.
\item Administrator jest odpowiedzialny za rozpatrywanie wniosków urlopowych nauczycieli. Może wyrazić zgodę na urlop lub odrzucić wniosek. Wtedy Nauczyciel otrzymuje stosowny komunikat o zmianie statusu wniosku.
\item Administrator ma pełny dostęp do edycji kalendarza szkolnego, klas, kół naukowych.
\item System oferuje pomoc przy wyborze zastępstwa za nauczyciela:
W przypadku, gdy nauczyciel przechodzi na urlop, system proponuje administratorowi możliwe zastępstwa godzin lekcyjnych nauczyciela przez innych ,,wolnych'' w tym momencie nauczycieli.
\item Administrator ma dostęp do danych nauczycieli, również tych niedostępnych z innych części systemu:
\begin{itemize}
\item szczegółowe dane personalne,
\item dane kontaktowe,
\item aktualne wynagrodzenie,
\item stanowisko,
\item staż pracy,
\item typ umowy,
\item ukończone kursy i szkolenia,
\item badania lekarskie,
\item wykorzystany urlop.
\end{itemize}
\end{enumerate}


% dostarczenie bazy danych zawierającej informacje o planie zajęć, kołach naukowych oraz uroczystościach szkolnych dostarczonych przez Zamawiającego w formie elektronicznej zgodnej z formatem pliku *.xls;
\subsection{Wymagania niefunkcjonalne}
\begin{enumerate}
	\item System powinien obsługiwać do 5000 użytkowników wszystkich typów kont.
	\item System powinien obsłużyć do 1000 jednoczesnych użytkowników bez zaistnienia opóźnień większych niż:
	\begin{enumerate}
		\item 	do 10 użytkowników 3s
		\item do 100 użytkowników 5s
		\item do 1000 użytkowników 10s
	\end{enumerate}
	\item W przypadku większej liczby użytkowników podłączonych równocześnie system powinien zachować stabilność lecz może zwiększyć opóźnienia
	\item Aplikacja webowa będzie napisana z wykorzystaniem nowoczesnych technologii HTML5, CSS3, JS i będzie kompatybilna z przeglądarkami:
	\begin{enumerate}
		\item Google Chrome $48>$
		\item Firefox $44>$
		\item Safari $9>$
		\item Internet Explorer $10>$
	\end{enumerate}
	\item Aplikacja powinna zostać zaimplementowana w technologiach, które można uruchomić na Serwerze systemem Windows Server 2012
	\item Szata graficzna powinna być podobna do systemu USOS
	\item System powinien spełniać standardy WCAG2.0
	\item System powinien być zintegrowany z systemem PayU, PayPal, eService w celu dokonywania płatności
	\item System powinien automatycznie wykonywać backup’y bazy danych codziennie pomiędzy godziną 2:00-5:00
	\item System powinien być tworzony zgodnie ze standardami przyrostowego sposobu  wytwarzania oprogramowania
	\item System powinien być prosty i intuicyjny w użyciu
	\item Maksymalne wymagania systemowe Systemu:
	\begin{enumerate}
		\item 64GB RAM
		\item 1TB pamięci dyskowej SSD
		\item moc obliczeniowa 10xCPU Intel Penitum II Xeon
	\end{enumerate}

	
\end{enumerate}


\subsection{Wymagania w zakresie gwarancji Systemu}
\begin{enumerate}
	\item Gwarancja świadczona przez Wykonawcę będzie dotyczyć Systemu, który opisany jest w niniejszym dokumencie.
	\item Wykonawca zobowiązuję się do przyjmowania zgłoszeń o awarii przez 7 dni w tygodniu, 24h na dobę.
	\item Czas reakcji serwisowej wynosi 4 godziny od otrzymania zgłoszenia od Zamawiającego.
	\item Wykonawca zobowiązuję się usunąć awarię w następującym czasie od momentu zgłoszenia przez Zamawiającego:
	\begin{enumerate}
		\item dla błędów krytycznych - \textbf{12h}
		\item dla błędów ważnych - \textbf{48h} 
		\item dla błędów zwykłych - \textbf{7dni}
	\end{enumerate}

	\item Priorytet błędu ustalany jest przez zamawiającego w dokumencie zgłoszenia o wystąpieniu błędu, zgodnie z definicjami:
	\begin{enumerate}
		\item \textbf{błąd krytyczny} - powoduje zatrzymanie pracy Systemu, bądź ma krytyczny wpływ na jego funkcjonalność,
		\item \textbf{błąd ważny} - powoduje poważne utrudnienia w pracy Systemu; powtarzające się zakłócenia
		\item \textbf{błąd zwykły} - problem, który nie powoduje bezpośredniego wpływu na funkcjonalność systemu.
	\end{enumerate}

	\item Zaplanowane przerwy działania systemu mogą mieć miejsce w dni wolne od pracy lub w dni robocze w godzinach 1:00-6:00. Każda planowana przerwa w systemie powinna zostać zgłoszona Zamawiającemu z wyprzedzeniem 4 dni.
	
\end{enumerate}
\subsection{Sposób wdrożenia}
% Łukasz
\paragraph{Etapy}
\begin{enumerate}
\item 
\end{enumerate}
\paragraph{Przebieg szkoleń}
W ramach wdrożenia systemu odbędzie się cykl szkoleń przeznaczonych dla uczniów, rodziców, nauczycieli i administratorów. Szkolenia będą prowadzone w salach szkolnych przeznaczonych do nauki informatyki - wyposażone w stanowiska komputerowe, projektor oraz tablicę do rysowania.
\begin{itemize}
\item Szkolenia dla uczniów odbywają się w ramach przedmiotu ,,informatyka''.
\item Szkolenia dla chętnych rodziców odbywają się raz na grupę osób w godzinach 17:00 - 19:00.
\item Szkolenia dla pracowników szkoły odbywają się w godzinach 09:00-12:00.
\end{itemize}
Pracownicy oddelegowani do udziału w szkoleniach obowiązani są uczestniczyć w nich w pełnym wymiarze czasu. Po zakończeniu szkoleń wszyscy uczestnicy wypełniają ,,Ankietę oceny szkolenia''.

Do systemu dołączony zostanie ,,samouczek'' - dokument opisujący działanie poszczególnych części systemu.

\subsection{Warunki odbioru systemu}
\section{Utrzymanie systemu po wdrożeniu}
odporność na błędy, że działa przez jakiś okres czasu
%TODO warunki co do ludzi utrzymujących: wykształcenie, certyfikaty

\section{Dokumentacja systemu}
% Pzypadkiem zrobiliśmy też to
\paragraph{Dokumentacja użytkownika} komplet dostarczonej dokumentacji będzie zawierać podręczniki dla użytkowników systemu zgodnie ze zdefiniowanymi w Systemie rolami. Podręcznik będzie zawierał wykaz czynności wykonywanych przez użytkownika pełniącego ustaloną rolę oraz szczegółowy sposób realizacji tych czynności (kolejne kroki) wraz ze zrzutami ekranów. Dokumentacja zostanie dostarczona w formie elektronicznej umożliwiającej wydruk.
\paragraph{Dokumentacja administratora}  w skład dokumentacji technicznej administratora wejdą dokumenty dotyczące następujących zagadnień: 
\begin{enumerate}
	\item użyte w projekcie oprogramowanie systemowe i narzędziowe, ze wskazaniem wersji oraz sposobu konfiguracji
	\item lista wykorzystanych bibliotek wraz ze wskazaniem wersji; 
	\item sposób instalacji i konfiguracji wszystkich składników sprzętu i oprogramowania; 
	\item dokumentacja struktur baz danych oraz konfiguracji poszczególnych elementów: serwerów, urządzeń sieciowych, aplikacji. 
	\item procedury tworzenia kopii i odtwarzania poszczególnych elementów Systemu.
\end{enumerate}
\paragraph{Testy akceptacyjne} zostaną opracowane przez Wykonawcę. Plan i scenariusze testów będą zgodne z powszechnie stosowanymu zasadami i praktykami. Przygotowany plan testów będzie określał:
\begin{enumerate}
	\item zasady przeprowadzania testów
	\item kolejność wykonywania scenariuszy testowych
	\item kryteria sukcesu dla poszczególnych scenariuszy
\end{enumerate}
Scenariusze będą pokrywały wszystkie wyspecyfikowane funkcje systemu. Każdy scenariusz określać będzie: dane, które muszą być wprowadzone do Systemu przed uruchomieniem scenariusza; kolejność czynności, wykonywanych w czasie testu oraz dane, wprowadzane do Systemu w czasie testu; oczekiwaną reakcję Systemu na wykonane czynności i wprowadzone dane. Dane testowe (do przeprowadzenia testów akceptacyjnych) w tym wszelkie materiały eksploatacyjne dostarczone będą przez Wykonawcę. 

\section{Zasoby}
Oświadczamy, iż dysponujemy następującymi zasobami:
\begin{itemize}
    \item Pliki w formacie *.xls zawierające wszystkie plany zajęć klas oraz plany zajęć nauczycieli
    \item Pliki w formacie *.xls zawierające spis uczniów z informacjami: imię, nazwisko, data urodzenia, pesel, adres, data rozpoczęcia szkoły, klasa, do której uczeń jest zapisany
    \item Pliki w formacie *.xls zawierające spis opiekunów uczniów z informacjami: imię i nazwisko opiekuna, pesel ucznia, adres, telefon kontaktowy 
    \item Pliki w formacie *.xls zawierające spis nauczycieli z informacjami: imię, nazwisko, adres, telefon kontaktowy, konto bankowe %TODO czy uwzględniamy wypłaty nauczycieli?
\end{itemize}

Wykonujący musi dysponować następującymi zasobami osobowymi:
\begin{itemize}
    \item co najmniej dwóch specjalistów baz danych o nie mniej niż dwuletnim doświadczeniu na podobnym stanowisku
    \item co najmniej czterech programistów posiadających certyfikat potwierdzający ich umiejętności programowania w języku Fortran
\end{itemize}


\section{Harmonogram wykonania Umowy}
\begin{enumerate}
    \item Umowa będzie realizowana od daty jej zawarcia i w okresie nie dłuższym niż do dnia 30 czerwca 2018 r., zgodnie ze szczegółowym Harmonogramem
    \item Wykonawca w terminie do 10 dni od dnia zawarcia Umowy, wytworzy i dostarczy do akceptacji Zamawiającego Harmonogram uwzględniający terminy i etapy zawarte w Umowie. Wykonawca uwzględni uwagi Zamawiającego dotyczące opracowanego Harmonogramu prac
    \item Wykonawca jest zobowiązany do realizacji przedmiotu Umowy w sposób następujący:
    \begin{enumerate}
        \item etap 1 - analiza przedwdrożeniowa - opracowanie projektu technicznego systemu eSzkoła w terminie do 30 dni od dnia zawarcia Umowy
        \item etap 2 - implementacja systemu w terminie do 275 dni od dnia zawarcia      Umowy nie później jednak niż do dnia 20 czerwca 2017 r.
        \item etap 3 - integracja systemu eSzkoła z istniejącymi rozwiązaniami dzienniczków elektronicznych LIBRUS oraz VULCAN w terminie nie późniejszym niż do dnia 20 czerwca 2017 r.
        \item Po realizacji etapów 1, 2 i 3 Wykonawca zobowiązuje się do wsparcia technicznego oraz zapewnienia usług serwisowych przez okres roku nie później niż do 30 czerwca 2018 r.
    \end{enumerate}
\end{enumerate}

\section{Kary umowne}
\end{document}
